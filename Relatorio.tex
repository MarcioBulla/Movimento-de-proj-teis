\documentclass[12pt]{article}

    \usepackage[breakable]{tcolorbox}
    \usepackage{parskip} % Stop auto-indenting (to mimic markdown behaviour)
    \usepackage[brazil]{babel}
    \usepackage{iftex}
    \usepackage{indentfirst}

    
    \everymath{\displaystyle}
    \ifPDFTeX
    	\usepackage[T1]{fontenc}
    	\usepackage{mathpazo}
    \else
    	\usepackage{fontspec}
    \fi

    % Basic figure setup, for now with no caption control since it's done
    % automatically by Pandoc (which extracts ![](path) syntax from Markdown).
    \usepackage{graphicx}
    % Maintain compatibility with old templates. Remove in nbconvert 6.0
    \let\Oldincludegraphics\includegraphics
    % Ensure that by default, figures have no caption (until we provide a
    % proper Figure object with a Caption API and a way to capture that
    % in the conversion process - todo).
    \usepackage{caption}
    \DeclareCaptionFormat{nocaption}{}
    \captionsetup{format=nocaption,aboveskip=0pt,belowskip=0pt}

    \usepackage{float}
    \floatplacement{figure}{H} % forces figures to be placed at the correct location
    \usepackage{xcolor} % Allow colors to be defined
    \usepackage{enumerate} % Needed for markdown enumerations to work
    \usepackage{geometry} % Used to adjust the document margins
    \usepackage{amsmath} % Equations
    \usepackage{amssymb} % Equations
    \usepackage{textcomp} % defines textquotesingle
    % Hack from http://tex.stackexchange.com/a/47451/13684:
    \AtBeginDocument{%
        \def\PYZsq{\textquotesingle}% Upright quotes in Pygmentized code
    }
    \usepackage{upquote} % Upright quotes for verbatim code
    \usepackage{eurosym} % defines \euro
    \usepackage[mathletters]{ucs} % Extended unicode (utf-8) support
    \usepackage{fancyvrb} % verbatim replacement that allows latex
    \usepackage{grffile} % extends the file name processing of package graphics 
                         % to support a larger range
    \makeatletter % fix for old versions of grffile with XeLaTeX
    \@ifpackagelater{grffile}{2019/11/01}
    {
      % Do nothing on new versions
    }
    {
      \def\Gread@@xetex#1{%
        \IfFileExists{"\Gin@base".bb}%
        {\Gread@eps{\Gin@base.bb}}%
        {\Gread@@xetex@aux#1}%
      }
    }
    \makeatother
    \usepackage[Export]{adjustbox} % Used to constrain images to a maximum size
    \adjustboxset{max size={0.9\linewidth}{0.9\paperheight}}

    % The hyperref package gives us a pdf with properly built
    % internal navigation ('pdf bookmarks' for the table of contents,
    % internal cross-reference links, web links for URLs, etc.)
    \usepackage{hyperref}
    % The default LaTeX title has an obnoxious amount of whitespace. By default,
    % titling removes some of it. It also provides customization options.
    \usepackage{titling}
    \usepackage{longtable} % longtable support required by pandoc >1.10
    \usepackage{booktabs}  % table support for pandoc > 1.12.2
    \usepackage[inline]{enumitem} % IRkernel/repr support (it uses the enumerate* environment)
    \usepackage[normalem]{ulem} % ulem is needed to support strikethroughs (\sout)
                                % normalem makes italics be italics, not underlines
    \usepackage{mathrsfs}
    

    
    % Colors for the hyperref package
    \definecolor{urlcolor}{rgb}{0,.145,.698}
    \definecolor{linkcolor}{rgb}{.71,0.21,0.01}
    \definecolor{citecolor}{rgb}{.12,.54,.11}

    % ANSI colors
    \definecolor{ansi-black}{HTML}{3E424D}
    \definecolor{ansi-black-intense}{HTML}{282C36}
    \definecolor{ansi-red}{HTML}{E75C58}
    \definecolor{ansi-red-intense}{HTML}{B22B31}
    \definecolor{ansi-green}{HTML}{00A250}
    \definecolor{ansi-green-intense}{HTML}{007427}
    \definecolor{ansi-yellow}{HTML}{DDB62B}
    \definecolor{ansi-yellow-intense}{HTML}{B27D12}
    \definecolor{ansi-blue}{HTML}{208FFB}
    \definecolor{ansi-blue-intense}{HTML}{0065CA}
    \definecolor{ansi-magenta}{HTML}{D160C4}
    \definecolor{ansi-magenta-intense}{HTML}{A03196}
    \definecolor{ansi-cyan}{HTML}{60C6C8}
    \definecolor{ansi-cyan-intense}{HTML}{258F8F}
    \definecolor{ansi-white}{HTML}{C5C1B4}
    \definecolor{ansi-white-intense}{HTML}{A1A6B2}
    \definecolor{ansi-default-inverse-fg}{HTML}{FFFFFF}
    \definecolor{ansi-default-inverse-bg}{HTML}{000000}

    % common color for the border for error outputs.
    \definecolor{outerrorbackground}{HTML}{FFDFDF}

    % commands and environments needed by pandoc snippets
    % extracted from the output of `pandoc -s`
    \providecommand{\tightlist}{%
      \setlength{\itemsep}{0pt}\setlength{\parskip}{0pt}}
    \DefineVerbatimEnvironment{Highlighting}{Verbatim}{commandchars=\\\{\}}
    % Add ',fontsize=\small' for more characters per line
    \newenvironment{Shaded}{}{}
    \newcommand{\KeywordTok}[1]{\textcolor[rgb]{0.00,0.44,0.13}{\textbf{{#1}}}}
    \newcommand{\DataTypeTok}[1]{\textcolor[rgb]{0.56,0.13,0.00}{{#1}}}
    \newcommand{\DecValTok}[1]{\textcolor[rgb]{0.25,0.63,0.44}{{#1}}}
    \newcommand{\BaseNTok}[1]{\textcolor[rgb]{0.25,0.63,0.44}{{#1}}}
    \newcommand{\FloatTok}[1]{\textcolor[rgb]{0.25,0.63,0.44}{{#1}}}
    \newcommand{\CharTok}[1]{\textcolor[rgb]{0.25,0.44,0.63}{{#1}}}
    \newcommand{\StringTok}[1]{\textcolor[rgb]{0.25,0.44,0.63}{{#1}}}
    \newcommand{\CommentTok}[1]{\textcolor[rgb]{0.38,0.63,0.69}{\textit{{#1}}}}
    \newcommand{\OtherTok}[1]{\textcolor[rgb]{0.00,0.44,0.13}{{#1}}}
    \newcommand{\AlertTok}[1]{\textcolor[rgb]{1.00,0.00,0.00}{\textbf{{#1}}}}
    \newcommand{\FunctionTok}[1]{\textcolor[rgb]{0.02,0.16,0.49}{{#1}}}
    \newcommand{\RegionMarkerTok}[1]{{#1}}
    \newcommand{\ErrorTok}[1]{\textcolor[rgb]{1.00,0.00,0.00}{\textbf{{#1}}}}
    \newcommand{\NormalTok}[1]{{#1}}
    
    % Additional commands for more recent versions of Pandoc
    \newcommand{\ConstantTok}[1]{\textcolor[rgb]{0.53,0.00,0.00}{{#1}}}
    \newcommand{\SpecialCharTok}[1]{\textcolor[rgb]{0.25,0.44,0.63}{{#1}}}
    \newcommand{\VerbatimStringTok}[1]{\textcolor[rgb]{0.25,0.44,0.63}{{#1}}}
    \newcommand{\SpecialStringTok}[1]{\textcolor[rgb]{0.73,0.40,0.53}{{#1}}}
    \newcommand{\ImportTok}[1]{{#1}}
    \newcommand{\DocumentationTok}[1]{\textcolor[rgb]{0.73,0.13,0.13}{\textit{{#1}}}}
    \newcommand{\AnnotationTok}[1]{\textcolor[rgb]{0.38,0.63,0.69}{\textbf{\textit{{#1}}}}}
    \newcommand{\CommentVarTok}[1]{\textcolor[rgb]{0.38,0.63,0.69}{\textbf{\textit{{#1}}}}}
    \newcommand{\VariableTok}[1]{\textcolor[rgb]{0.10,0.09,0.49}{{#1}}}
    \newcommand{\ControlFlowTok}[1]{\textcolor[rgb]{0.00,0.44,0.13}{\textbf{{#1}}}}
    \newcommand{\OperatorTok}[1]{\textcolor[rgb]{0.40,0.40,0.40}{{#1}}}
    \newcommand{\BuiltInTok}[1]{{#1}}
    \newcommand{\ExtensionTok}[1]{{#1}}
    \newcommand{\PreprocessorTok}[1]{\textcolor[rgb]{0.74,0.48,0.00}{{#1}}}
    \newcommand{\AttributeTok}[1]{\textcolor[rgb]{0.49,0.56,0.16}{{#1}}}
    \newcommand{\InformationTok}[1]{\textcolor[rgb]{0.38,0.63,0.69}{\textbf{\textit{{#1}}}}}
    \newcommand{\WarningTok}[1]{\textcolor[rgb]{0.38,0.63,0.69}{\textbf{\textit{{#1}}}}}
    
    
    % Define a nice break command that doesn't care if a line doesn't already
    % exist.
    \def\br{\hspace*{\fill} \\* }
    % Math Jax compatibility definitions
    \def\gt{>}
    \def\lt{<}
    \let\Oldtex\TeX
    \let\Oldlatex\LaTeX
    \renewcommand{\TeX}{\textrm{\Oldtex}}
    \renewcommand{\LaTeX}{\textrm{\Oldlatex}}
    % Document parameters
    % Document title
\title{Relatório Mecânica Clássica}
\author{Marcio Ap. Bulla Junior\\
\small Universidade Federal do Paraná - Centro Politécnico - Jardim das Américas -\\
    \small 81531-980 - Curitiba - PR - Brasil}
\date{\today}
    
    
    
% Pygments definitions
\makeatletter
\def\PY@reset{\let\PY@it=\relax \let\PY@bf=\relax%
    \let\PY@ul=\relax \let\PY@tc=\relax%
    \let\PY@bc=\relax \let\PY@ff=\relax}
\def\PY@tok#1{\csname PY@tok@#1\endcsname}
\def\PY@toks#1+{\ifx\relax#1\empty\else%
    \PY@tok{#1}\expandafter\PY@toks\fi}
\def\PY@do#1{\PY@bc{\PY@tc{\PY@ul{%
    \PY@it{\PY@bf{\PY@ff{#1}}}}}}}
\def\PY#1#2{\PY@reset\PY@toks#1+\relax+\PY@do{#2}}

\expandafter\def\csname PY@tok@w\endcsname{\def\PY@tc##1{\textcolor[rgb]{0.73,0.73,0.73}{##1}}}
\expandafter\def\csname PY@tok@c\endcsname{\let\PY@it=\textit\def\PY@tc##1{\textcolor[rgb]{0.25,0.50,0.50}{##1}}}
\expandafter\def\csname PY@tok@cp\endcsname{\def\PY@tc##1{\textcolor[rgb]{0.74,0.48,0.00}{##1}}}
\expandafter\def\csname PY@tok@k\endcsname{\let\PY@bf=\textbf\def\PY@tc##1{\textcolor[rgb]{0.00,0.50,0.00}{##1}}}
\expandafter\def\csname PY@tok@kp\endcsname{\def\PY@tc##1{\textcolor[rgb]{0.00,0.50,0.00}{##1}}}
\expandafter\def\csname PY@tok@kt\endcsname{\def\PY@tc##1{\textcolor[rgb]{0.69,0.00,0.25}{##1}}}
\expandafter\def\csname PY@tok@o\endcsname{\def\PY@tc##1{\textcolor[rgb]{0.40,0.40,0.40}{##1}}}
\expandafter\def\csname PY@tok@ow\endcsname{\let\PY@bf=\textbf\def\PY@tc##1{\textcolor[rgb]{0.67,0.13,1.00}{##1}}}
\expandafter\def\csname PY@tok@nb\endcsname{\def\PY@tc##1{\textcolor[rgb]{0.00,0.50,0.00}{##1}}}
\expandafter\def\csname PY@tok@nf\endcsname{\def\PY@tc##1{\textcolor[rgb]{0.00,0.00,1.00}{##1}}}
\expandafter\def\csname PY@tok@nc\endcsname{\let\PY@bf=\textbf\def\PY@tc##1{\textcolor[rgb]{0.00,0.00,1.00}{##1}}}
\expandafter\def\csname PY@tok@nn\endcsname{\let\PY@bf=\textbf\def\PY@tc##1{\textcolor[rgb]{0.00,0.00,1.00}{##1}}}
\expandafter\def\csname PY@tok@ne\endcsname{\let\PY@bf=\textbf\def\PY@tc##1{\textcolor[rgb]{0.82,0.25,0.23}{##1}}}
\expandafter\def\csname PY@tok@nv\endcsname{\def\PY@tc##1{\textcolor[rgb]{0.10,0.09,0.49}{##1}}}
\expandafter\def\csname PY@tok@no\endcsname{\def\PY@tc##1{\textcolor[rgb]{0.53,0.00,0.00}{##1}}}
\expandafter\def\csname PY@tok@nl\endcsname{\def\PY@tc##1{\textcolor[rgb]{0.63,0.63,0.00}{##1}}}
\expandafter\def\csname PY@tok@ni\endcsname{\let\PY@bf=\textbf\def\PY@tc##1{\textcolor[rgb]{0.60,0.60,0.60}{##1}}}
\expandafter\def\csname PY@tok@na\endcsname{\def\PY@tc##1{\textcolor[rgb]{0.49,0.56,0.16}{##1}}}
\expandafter\def\csname PY@tok@nt\endcsname{\let\PY@bf=\textbf\def\PY@tc##1{\textcolor[rgb]{0.00,0.50,0.00}{##1}}}
\expandafter\def\csname PY@tok@nd\endcsname{\def\PY@tc##1{\textcolor[rgb]{0.67,0.13,1.00}{##1}}}
\expandafter\def\csname PY@tok@s\endcsname{\def\PY@tc##1{\textcolor[rgb]{0.73,0.13,0.13}{##1}}}
\expandafter\def\csname PY@tok@sd\endcsname{\let\PY@it=\textit\def\PY@tc##1{\textcolor[rgb]{0.73,0.13,0.13}{##1}}}
\expandafter\def\csname PY@tok@si\endcsname{\let\PY@bf=\textbf\def\PY@tc##1{\textcolor[rgb]{0.73,0.40,0.53}{##1}}}
\expandafter\def\csname PY@tok@se\endcsname{\let\PY@bf=\textbf\def\PY@tc##1{\textcolor[rgb]{0.73,0.40,0.13}{##1}}}
\expandafter\def\csname PY@tok@sr\endcsname{\def\PY@tc##1{\textcolor[rgb]{0.73,0.40,0.53}{##1}}}
\expandafter\def\csname PY@tok@ss\endcsname{\def\PY@tc##1{\textcolor[rgb]{0.10,0.09,0.49}{##1}}}
\expandafter\def\csname PY@tok@sx\endcsname{\def\PY@tc##1{\textcolor[rgb]{0.00,0.50,0.00}{##1}}}
\expandafter\def\csname PY@tok@m\endcsname{\def\PY@tc##1{\textcolor[rgb]{0.40,0.40,0.40}{##1}}}
\expandafter\def\csname PY@tok@gh\endcsname{\let\PY@bf=\textbf\def\PY@tc##1{\textcolor[rgb]{0.00,0.00,0.50}{##1}}}
\expandafter\def\csname PY@tok@gu\endcsname{\let\PY@bf=\textbf\def\PY@tc##1{\textcolor[rgb]{0.50,0.00,0.50}{##1}}}
\expandafter\def\csname PY@tok@gd\endcsname{\def\PY@tc##1{\textcolor[rgb]{0.63,0.00,0.00}{##1}}}
\expandafter\def\csname PY@tok@gi\endcsname{\def\PY@tc##1{\textcolor[rgb]{0.00,0.63,0.00}{##1}}}
\expandafter\def\csname PY@tok@gr\endcsname{\def\PY@tc##1{\textcolor[rgb]{1.00,0.00,0.00}{##1}}}
\expandafter\def\csname PY@tok@ge\endcsname{\let\PY@it=\textit}
\expandafter\def\csname PY@tok@gs\endcsname{\let\PY@bf=\textbf}
\expandafter\def\csname PY@tok@gp\endcsname{\let\PY@bf=\textbf\def\PY@tc##1{\textcolor[rgb]{0.00,0.00,0.50}{##1}}}
\expandafter\def\csname PY@tok@go\endcsname{\def\PY@tc##1{\textcolor[rgb]{0.53,0.53,0.53}{##1}}}
\expandafter\def\csname PY@tok@gt\endcsname{\def\PY@tc##1{\textcolor[rgb]{0.00,0.27,0.87}{##1}}}
\expandafter\def\csname PY@tok@err\endcsname{\def\PY@bc##1{\setlength{\fboxsep}{0pt}\fcolorbox[rgb]{1.00,0.00,0.00}{1,1,1}{\strut ##1}}}
\expandafter\def\csname PY@tok@kc\endcsname{\let\PY@bf=\textbf\def\PY@tc##1{\textcolor[rgb]{0.00,0.50,0.00}{##1}}}
\expandafter\def\csname PY@tok@kd\endcsname{\let\PY@bf=\textbf\def\PY@tc##1{\textcolor[rgb]{0.00,0.50,0.00}{##1}}}
\expandafter\def\csname PY@tok@kn\endcsname{\let\PY@bf=\textbf\def\PY@tc##1{\textcolor[rgb]{0.00,0.50,0.00}{##1}}}
\expandafter\def\csname PY@tok@kr\endcsname{\let\PY@bf=\textbf\def\PY@tc##1{\textcolor[rgb]{0.00,0.50,0.00}{##1}}}
\expandafter\def\csname PY@tok@bp\endcsname{\def\PY@tc##1{\textcolor[rgb]{0.00,0.50,0.00}{##1}}}
\expandafter\def\csname PY@tok@fm\endcsname{\def\PY@tc##1{\textcolor[rgb]{0.00,0.00,1.00}{##1}}}
\expandafter\def\csname PY@tok@vc\endcsname{\def\PY@tc##1{\textcolor[rgb]{0.10,0.09,0.49}{##1}}}
\expandafter\def\csname PY@tok@vg\endcsname{\def\PY@tc##1{\textcolor[rgb]{0.10,0.09,0.49}{##1}}}
\expandafter\def\csname PY@tok@vi\endcsname{\def\PY@tc##1{\textcolor[rgb]{0.10,0.09,0.49}{##1}}}
\expandafter\def\csname PY@tok@vm\endcsname{\def\PY@tc##1{\textcolor[rgb]{0.10,0.09,0.49}{##1}}}
\expandafter\def\csname PY@tok@sa\endcsname{\def\PY@tc##1{\textcolor[rgb]{0.73,0.13,0.13}{##1}}}
\expandafter\def\csname PY@tok@sb\endcsname{\def\PY@tc##1{\textcolor[rgb]{0.73,0.13,0.13}{##1}}}
\expandafter\def\csname PY@tok@sc\endcsname{\def\PY@tc##1{\textcolor[rgb]{0.73,0.13,0.13}{##1}}}
\expandafter\def\csname PY@tok@dl\endcsname{\def\PY@tc##1{\textcolor[rgb]{0.73,0.13,0.13}{##1}}}
\expandafter\def\csname PY@tok@s2\endcsname{\def\PY@tc##1{\textcolor[rgb]{0.73,0.13,0.13}{##1}}}
\expandafter\def\csname PY@tok@sh\endcsname{\def\PY@tc##1{\textcolor[rgb]{0.73,0.13,0.13}{##1}}}
\expandafter\def\csname PY@tok@s1\endcsname{\def\PY@tc##1{\textcolor[rgb]{0.73,0.13,0.13}{##1}}}
\expandafter\def\csname PY@tok@mb\endcsname{\def\PY@tc##1{\textcolor[rgb]{0.40,0.40,0.40}{##1}}}
\expandafter\def\csname PY@tok@mf\endcsname{\def\PY@tc##1{\textcolor[rgb]{0.40,0.40,0.40}{##1}}}
\expandafter\def\csname PY@tok@mh\endcsname{\def\PY@tc##1{\textcolor[rgb]{0.40,0.40,0.40}{##1}}}
\expandafter\def\csname PY@tok@mi\endcsname{\def\PY@tc##1{\textcolor[rgb]{0.40,0.40,0.40}{##1}}}
\expandafter\def\csname PY@tok@il\endcsname{\def\PY@tc##1{\textcolor[rgb]{0.40,0.40,0.40}{##1}}}
\expandafter\def\csname PY@tok@mo\endcsname{\def\PY@tc##1{\textcolor[rgb]{0.40,0.40,0.40}{##1}}}
\expandafter\def\csname PY@tok@ch\endcsname{\let\PY@it=\textit\def\PY@tc##1{\textcolor[rgb]{0.25,0.50,0.50}{##1}}}
\expandafter\def\csname PY@tok@cm\endcsname{\let\PY@it=\textit\def\PY@tc##1{\textcolor[rgb]{0.25,0.50,0.50}{##1}}}
\expandafter\def\csname PY@tok@cpf\endcsname{\let\PY@it=\textit\def\PY@tc##1{\textcolor[rgb]{0.25,0.50,0.50}{##1}}}
\expandafter\def\csname PY@tok@c1\endcsname{\let\PY@it=\textit\def\PY@tc##1{\textcolor[rgb]{0.25,0.50,0.50}{##1}}}
\expandafter\def\csname PY@tok@cs\endcsname{\let\PY@it=\textit\def\PY@tc##1{\textcolor[rgb]{0.25,0.50,0.50}{##1}}}

\def\PYZbs{\char`\\}
\def\PYZus{\char`\_}
\def\PYZob{\char`\{}
\def\PYZcb{\char`\}}
\def\PYZca{\char`\^}
\def\PYZam{\char`\&}
\def\PYZlt{\char`\<}
\def\PYZgt{\char`\>}
\def\PYZsh{\char`\#}
\def\PYZpc{\char`\%}
\def\PYZdl{\char`\$}
\def\PYZhy{\char`\-}
\def\PYZsq{\char`\'}
\def\PYZdq{\char`\"}
\def\PYZti{\char`\~}
% for compatibility with earlier versions
\def\PYZat{@}
\def\PYZlb{[}
\def\PYZrb{]}
\makeatother


    % For linebreaks inside Verbatim environment from package fancyvrb. 
    \makeatletter
        \newbox\Wrappedcontinuationbox 
        \newbox\Wrappedvisiblespacebox 
        \newcommand*\Wrappedvisiblespace {\textcolor{red}{\textvisiblespace}} 
        \newcommand*\Wrappedcontinuationsymbol {\textcolor{red}{\llap{\tiny$\m@th\hookrightarrow$}}} 
        \newcommand*\Wrappedcontinuationindent {3ex } 
        \newcommand*\Wrappedafterbreak {\kern\Wrappedcontinuationindent\copy\Wrappedcontinuationbox} 
        % Take advantage of the already applied Pygments mark-up to insert 
        % potential linebreaks for TeX processing. 
        %        {, <, #, %, $, ' and ": go to next line. 
        %        _, }, ^, &, >, - and ~: stay at end of broken line. 
        % Use of \textquotesingle for straight quote. 
        \newcommand*\Wrappedbreaksatspecials {% 
            \def\PYGZus{\discretionary{\char`\_}{\Wrappedafterbreak}{\char`\_}}% 
            \def\PYGZob{\discretionary{}{\Wrappedafterbreak\char`\{}{\char`\{}}% 
            \def\PYGZcb{\discretionary{\char`\}}{\Wrappedafterbreak}{\char`\}}}% 
            \def\PYGZca{\discretionary{\char`\^}{\Wrappedafterbreak}{\char`\^}}% 
            \def\PYGZam{\discretionary{\char`\&}{\Wrappedafterbreak}{\char`\&}}% 
            \def\PYGZlt{\discretionary{}{\Wrappedafterbreak\char`\<}{\char`\<}}% 
            \def\PYGZgt{\discretionary{\char`\>}{\Wrappedafterbreak}{\char`\>}}% 
            \def\PYGZsh{\discretionary{}{\Wrappedafterbreak\char`\#}{\char`\#}}% 
            \def\PYGZpc{\discretionary{}{\Wrappedafterbreak\char`\%}{\char`\%}}% 
            \def\PYGZdl{\discretionary{}{\Wrappedafterbreak\char`\$}{\char`\$}}% 
            \def\PYGZhy{\discretionary{\char`\-}{\Wrappedafterbreak}{\char`\-}}% 
            \def\PYGZsq{\discretionary{}{\Wrappedafterbreak\textquotesingle}{\textquotesingle}}% 
            \def\PYGZdq{\discretionary{}{\Wrappedafterbreak\char`\"}{\char`\"}}% 
            \def\PYGZti{\discretionary{\char`\~}{\Wrappedafterbreak}{\char`\~}}% 
        } 
        % Some characters . , ; ? ! / are not pygmentized. 
        % This macro makes them "active" and they will insert potential linebreaks 
        \newcommand*\Wrappedbreaksatpunct {% 
            \lccode`\~`\.\lowercase{\def~}{\discretionary{\hbox{\char`\.}}{\Wrappedafterbreak}{\hbox{\char`\.}}}% 
            \lccode`\~`\,\lowercase{\def~}{\discretionary{\hbox{\char`\,}}{\Wrappedafterbreak}{\hbox{\char`\,}}}% 
            \lccode`\~`\;\lowercase{\def~}{\discretionary{\hbox{\char`\;}}{\Wrappedafterbreak}{\hbox{\char`\;}}}% 
            \lccode`\~`\:\lowercase{\def~}{\discretionary{\hbox{\char`\:}}{\Wrappedafterbreak}{\hbox{\char`\:}}}% 
            \lccode`\~`\?\lowercase{\def~}{\discretionary{\hbox{\char`\?}}{\Wrappedafterbreak}{\hbox{\char`\?}}}% 
            \lccode`\~`\!\lowercase{\def~}{\discretionary{\hbox{\char`\!}}{\Wrappedafterbreak}{\hbox{\char`\!}}}% 
            \lccode`\~`\/\lowercase{\def~}{\discretionary{\hbox{\char`\/}}{\Wrappedafterbreak}{\hbox{\char`\/}}}% 
            \catcode`\.\active
            \catcode`\,\active 
            \catcode`\;\active
            \catcode`\:\active
            \catcode`\?\active
            \catcode`\!\active
            \catcode`\/\active 
            \lccode`\~`\~ 	
        }
    \makeatother

    \let\OriginalVerbatim=\Verbatim
    \makeatletter
    \renewcommand{\Verbatim}[1][1]{%
        %\parskip\z@skip
        \sbox\Wrappedcontinuationbox {\Wrappedcontinuationsymbol}%
        \sbox\Wrappedvisiblespacebox {\FV@SetupFont\Wrappedvisiblespace}%
        \def\FancyVerbFormatLine ##1{\hsize\linewidth
            \vtop{\raggedright\hyphenpenalty\z@\exhyphenpenalty\z@
                \doublehyphendemerits\z@\finalhyphendemerits\z@
                \strut ##1\strut}%
        }%
        % If the linebreak is at a space, the latter will be displayed as visible
        % space at end of first line, and a continuation symbol starts next line.
        % Stretch/shrink are however usually zero for typewriter font.
        \def\FV@Space {%
            \nobreak\hskip\z@ plus\fontdimen3\font minus\fontdimen4\font
            \discretionary{\copy\Wrappedvisiblespacebox}{\Wrappedafterbreak}
            {\kern\fontdimen2\font}%
        }%
        
        % Allow breaks at special characters using \PYG... macros.
        \Wrappedbreaksatspecials
        % Breaks at punctuation characters . , ; ? ! and / need catcode=\active 	
        \OriginalVerbatim[#1,codes*=\Wrappedbreaksatpunct]%
    }
    \makeatother

    % Exact colors from NB
    \definecolor{incolor}{HTML}{303F9F}
    \definecolor{outcolor}{HTML}{D84315}
    \definecolor{cellborder}{HTML}{CFCFCF}
    \definecolor{cellbackground}{HTML}{F7F7F7}
    
    % prompt
    \makeatletter
    \newcommand{\boxspacing}{\kern\kvtcb@left@rule\kern\kvtcb@boxsep}
    \makeatother
    \newcommand{\prompt}[4]{
        {\ttfamily\llap{{\color{#2}[#3]:\hspace{3pt}#4}}\vspace{-\baselineskip}}
    }
    

    
    % Prevent overflowing lines due to hard-to-break entities
    \sloppy 
    % Setup hyperref package
    \hypersetup{
      breaklinks=true,  % so long urls are correctly broken across lines
      colorlinks=true,
      urlcolor=urlcolor,
      linkcolor=linkcolor,
      citecolor=citecolor,
      }
    % Slightly bigger margins than the latex defaults
    
    \geometry{verbose,tmargin=1in,bmargin=1in,lmargin=1in,rmargin=1in}
    
    

\begin{document}
    
    \maketitle
    
    

    
    \hypertarget{metodologia}{%
\section*{metodologia}\label{metodologia}}

Usaremos o \emph{python} com o \emph{jupyter}, o qual facilita as
aplicações e visualizações dos dados. Dessa forma, para obter os
gráficos, precisamos encontrar como a velocidade e a posição varia com o
tempo. Com isto, para resolver as EDOs, utilizaremos o \emph{scipy},
\emph{numpy} e \emph{matplotlib} para gerar os gráficos. Além disso,
sempre usaremos valores na unidade internacional com exceção dos
radianos, pois facilita apresentar os dados em graus. Dessa forma, não
será informado durante todo o documento, pois já foi definido aqui.

Todo o código-fonte está no repositório do GitHub:
\url{https://github.com/MarcioBulla/Movimento-de-projeteis}

\hypertarget{importando-bibliotecas-mencionadas}{%
\subsection*{Importando bibliotecas
mencionadas:}\label{importando-bibliotecas-mencionadas}}

    \begin{tcolorbox}[breakable, size=fbox, boxrule=1pt, pad at break*=1mm,colback=cellbackground, colframe=cellborder]
\prompt{In}{incolor}{1}{\boxspacing}
\begin{Verbatim}[commandchars=\\\{\}]
\PY{k+kn}{import} \PY{n+nn}{numpy} \PY{k}{as} \PY{n+nn}{np}
\PY{k+kn}{from} \PY{n+nn}{scipy}\PY{n+nn}{.}\PY{n+nn}{integrate} \PY{k+kn}{import} \PY{n}{solve\PYZus{}ivp}
\PY{k+kn}{import} \PY{n+nn}{matplotlib}\PY{n+nn}{.}\PY{n+nn}{pyplot} \PY{k}{as} \PY{n+nn}{plt}
\PY{k+kn}{import} \PY{n+nn}{matplotlib} \PY{k}{as} \PY{n+nn}{mpl}
\end{Verbatim}
\end{tcolorbox}

    \hypertarget{definindo-parametros-para-plotar-os-graficos-de-forma-mais-atraente}{%
\subsubsection*{Definindo parametros para plotar os graficos de forma
mais
atraente:}\label{definindo-parametros-para-plotar-os-graficos-de-forma-mais-atraente}}

    \begin{tcolorbox}[breakable, size=fbox, boxrule=1pt, pad at break*=1mm,colback=cellbackground, colframe=cellborder]
\prompt{In}{incolor}{2}{\boxspacing}
\begin{Verbatim}[commandchars=\\\{\}]
\PY{n}{norm} \PY{o}{=} \PY{n}{mpl}\PY{o}{.}\PY{n}{colors}\PY{o}{.}\PY{n}{Normalize}\PY{p}{(}\PY{n}{vmin}\PY{o}{=}\PY{l+m+mi}{0}\PY{p}{,} \PY{n}{vmax}\PY{o}{=}\PY{l+m+mi}{90}\PY{p}{)}
\PY{n}{cmap} \PY{o}{=} \PY{n}{mpl}\PY{o}{.}\PY{n}{cm}\PY{o}{.}\PY{n}{ScalarMappable}\PY{p}{(}\PY{n}{norm}\PY{o}{=}\PY{n}{norm}\PY{p}{,} \PY{n}{cmap}\PY{o}{=}\PY{n}{mpl}\PY{o}{.}\PY{n}{cm}\PY{o}{.}\PY{n}{gist\PYZus{}rainbow}\PY{p}{)}
\PY{n}{cmap}\PY{o}{.}\PY{n}{set\PYZus{}array}\PY{p}{(}\PY{p}{[}\PY{p}{]}\PY{p}{)}
\PY{n}{cmap2} \PY{o}{=} \PY{n}{mpl}\PY{o}{.}\PY{n}{cm}\PY{o}{.}\PY{n}{ScalarMappable}\PY{p}{(}\PY{n}{norm}\PY{o}{=}\PY{n}{norm}\PY{p}{,} \PY{n}{cmap}\PY{o}{=}\PY{n}{mpl}\PY{o}{.}\PY{n}{cm}\PY{o}{.}\PY{n}{rainbow}\PY{p}{)}
\PY{n}{cmap2}\PY{o}{.}\PY{n}{set\PYZus{}array}\PY{p}{(}\PY{p}{[}\PY{p}{]}\PY{p}{)}
\PY{n}{plt}\PY{o}{.}\PY{n}{rcParams}\PY{p}{[}\PY{l+s+s2}{\PYZdq{}}\PY{l+s+s2}{xtick.labelsize}\PY{l+s+s2}{\PYZdq{}}\PY{p}{]} \PY{o}{=} \PY{l+m+mi}{15}
\PY{n}{plt}\PY{o}{.}\PY{n}{rcParams}\PY{p}{[}\PY{l+s+s2}{\PYZdq{}}\PY{l+s+s2}{ytick.labelsize}\PY{l+s+s2}{\PYZdq{}}\PY{p}{]} \PY{o}{=} \PY{l+m+mi}{15}
\PY{n}{plt}\PY{o}{.}\PY{n}{rcParams}\PY{p}{[}\PY{l+s+s2}{\PYZdq{}}\PY{l+s+s2}{axes.labelsize}\PY{l+s+s2}{\PYZdq{}}\PY{p}{]} \PY{o}{=} \PY{l+m+mi}{20}
\PY{n}{plt}\PY{o}{.}\PY{n}{rcParams}\PY{p}{[}\PY{l+s+s2}{\PYZdq{}}\PY{l+s+s2}{axes.titlesize}\PY{l+s+s2}{\PYZdq{}}\PY{p}{]} \PY{o}{=} \PY{l+m+mi}{25}
\PY{n}{plt}\PY{o}{.}\PY{n}{rcParams}\PY{p}{[}\PY{l+s+s2}{\PYZdq{}}\PY{l+s+s2}{figure.figsize}\PY{l+s+s2}{\PYZdq{}}\PY{p}{]} \PY{o}{=} \PY{p}{[}\PY{l+m+mi}{11}\PY{p}{,} \PY{l+m+mi}{7}\PY{p}{]}
\PY{n}{plt}\PY{o}{.}\PY{n}{rcParams}\PY{p}{[}\PY{l+s+s2}{\PYZdq{}}\PY{l+s+s2}{lines.linewidth}\PY{l+s+s2}{\PYZdq{}}\PY{p}{]} \PY{o}{=} \PY{l+m+mi}{3}
\PY{n}{plt}\PY{o}{.}\PY{n}{rcParams}\PY{p}{[}\PY{l+s+s2}{\PYZdq{}}\PY{l+s+s2}{axes.grid}\PY{l+s+s2}{\PYZdq{}}\PY{p}{]} \PY{o}{=} \PY{k+kc}{True}
\PY{n}{plt}\PY{o}{.}\PY{n}{rcParams}\PY{p}{[}\PY{l+s+s2}{\PYZdq{}}\PY{l+s+s2}{legend.fontsize}\PY{l+s+s2}{\PYZdq{}}\PY{p}{]} \PY{o}{=} \PY{l+m+mi}{15}
\PY{n}{plt}\PY{o}{.}\PY{n}{rcParams}\PY{p}{[}\PY{l+s+s2}{\PYZdq{}}\PY{l+s+s2}{figure.autolayout}\PY{l+s+s2}{\PYZdq{}}\PY{p}{]} \PY{o}{=} \PY{k+kc}{True}
\PY{n}{plt}\PY{o}{.}\PY{n}{rcParams}\PY{p}{[}\PY{l+s+s2}{\PYZdq{}}\PY{l+s+s2}{axes.xmargin}\PY{l+s+s2}{\PYZdq{}}\PY{p}{]} \PY{o}{=} \PY{l+m+mi}{0}
\PY{n}{plt}\PY{o}{.}\PY{n}{rcParams}\PY{p}{[}\PY{l+s+s2}{\PYZdq{}}\PY{l+s+s2}{axes.ymargin}\PY{l+s+s2}{\PYZdq{}}\PY{p}{]} \PY{o}{=} \PY{l+m+mi}{0}
\PY{n}{plt}\PY{o}{.}\PY{n}{rcParams}\PY{p}{[}\PY{l+s+s2}{\PYZdq{}}\PY{l+s+s2}{lines.markersize}\PY{l+s+s2}{\PYZdq{}}\PY{p}{]} \PY{o}{=} \PY{l+m+mi}{10}
\end{Verbatim}
\end{tcolorbox}

    \hypertarget{questuxe3o-1}{%
\section{Questão 1}\label{questuxe3o-1}}

As condições iniciais são: \(x(0)=z(0)=0\) e \(\vec{v}(0)=v_0\hat{v}\).

Sendo
\(\vec{v_0} = v_0\cos{\theta}\hat{\imath} + v_0\sin{\theta}\hat{k}\)

\hypertarget{alternativa-a}{%
\subsection{Alternativa (a)}\label{alternativa-a}}

Para resolver esta questão precisamos desmembrar o vetor posição em suas
componentes. \[ m \frac{d^2\vec{r}}{dt^2} =-mg\hat{k} - cv\vec{v}\]

Dessa forma, temos duas EDO de segunda ordem.

\[ \frac{d^2x}{dt^2} = - \frac{c}{m} \frac{dx}{dt}\sqrt{\left(\frac{dx}{dt} \right)^2 + \left(\frac{dz}{dt} \right)^2}
\]
\[\frac{d^2z}{dt^2} = - g - \frac{c}{m} \frac{dz}{dt}\sqrt{\left(\frac{dx}{dt} \right)^2 + \left(\frac{dz}{dt} \right)^2}
\]

Usaremos as função \texttt{solve\_ivp} do \emph{scipe} para encontrar as
soluções.

\hypertarget{definindo-a-funuxe7uxe3o-das-edos-para-resolvuxea-las}{%
\paragraph{Definindo a função das EDOs para
resolvê-las:}\label{definindo-a-funuxe7uxe3o-das-edos-para-resolvuxea-las}}

    \begin{tcolorbox}[breakable, size=fbox, boxrule=1pt, pad at break*=1mm,colback=cellbackground, colframe=cellborder]
\prompt{In}{incolor}{3}{\boxspacing}
\begin{Verbatim}[commandchars=\\\{\}]
\PY{k}{def} \PY{n+nf}{r}\PY{p}{(}\PY{n}{t}\PY{p}{,} \PY{n}{r}\PY{p}{,} \PY{n}{g}\PY{p}{,} \PY{n}{c}\PY{p}{,} \PY{n}{m}\PY{p}{)}\PY{p}{:}
    \PY{n}{x}\PY{p}{,} \PY{n}{z}\PY{p}{,} \PY{n}{vx}\PY{p}{,} \PY{n}{vz} \PY{o}{=} \PY{n}{r}
    \PY{n}{ddx} \PY{o}{=} \PY{o}{\PYZhy{}}\PY{n}{c}\PY{o}{/}\PY{n}{m} \PY{o}{*} \PY{n}{vx} \PY{o}{*} \PY{n}{np}\PY{o}{.}\PY{n}{hypot}\PY{p}{(}\PY{n}{x}\PY{p}{,} \PY{n}{z}\PY{p}{)}  \PY{c+c1}{\PYZsh{} Derivada da segunda de x}
    \PY{n}{ddz} \PY{o}{=} \PY{o}{\PYZhy{}}\PY{n}{g} \PY{o}{\PYZhy{}} \PY{n}{c}\PY{o}{/}\PY{n}{m} \PY{o}{*} \PY{n}{vz} \PY{o}{*} \PY{n}{np}\PY{o}{.}\PY{n}{hypot}\PY{p}{(}\PY{n}{x}\PY{p}{,} \PY{n}{z}\PY{p}{)}  \PY{c+c1}{\PYZsh{} Derivada da segunda de z}
    \PY{k}{return} \PY{p}{[}\PY{n}{vx}\PY{p}{,} \PY{n}{vz}\PY{p}{,} \PY{n}{ddx}\PY{p}{,} \PY{n}{ddz}\PY{p}{]}
\end{Verbatim}
\end{tcolorbox}

    Observe que, r é um vetor posição e o vetor velocidade concatenados.
Dessa forma, estamos separando as componentes deste vetor. Com isso,
definimos as EDOs que são as derivadas da segunda e, retornamos as EDOs
em ordem das derivadas.

\hypertarget{definindo-condiuxe7uxf5es-iniciais-constantes-intervalo-da-soluuxe7uxe3o-numuxe9rica}{%
\paragraph{Definindo condições iniciais, constantes, intervalo da
solução
numérica:}\label{definindo-condiuxe7uxf5es-iniciais-constantes-intervalo-da-soluuxe7uxe3o-numuxe9rica}}

    \begin{tcolorbox}[breakable, size=fbox, boxrule=1pt, pad at break*=1mm,colback=cellbackground, colframe=cellborder]
\prompt{In}{incolor}{4}{\boxspacing}
\begin{Verbatim}[commandchars=\\\{\}]
\PY{c+c1}{\PYZsh{} Condições iniciais}
\PY{n}{x0} \PY{o}{=} \PY{l+m+mi}{0}
\PY{n}{z0} \PY{o}{=} \PY{l+m+mi}{0}
\PY{n}{v0} \PY{o}{=} \PY{l+m+mi}{30}
\PY{n}{theta} \PY{o}{=} \PY{n}{np}\PY{o}{.}\PY{n}{radians}\PY{p}{(}\PY{l+m+mi}{45}\PY{p}{)}
\PY{n}{vx0} \PY{o}{=} \PY{n}{v0}\PY{o}{*}\PY{n}{np}\PY{o}{.}\PY{n}{cos}\PY{p}{(}\PY{n}{theta}\PY{p}{)}
\PY{n}{vz0} \PY{o}{=} \PY{n}{v0}\PY{o}{*}\PY{n}{np}\PY{o}{.}\PY{n}{sin}\PY{p}{(}\PY{n}{theta}\PY{p}{)}
\PY{c+c1}{\PYZsh{} Constantes}
\PY{n}{g} \PY{o}{=} \PY{l+m+mf}{9.8}
\PY{n}{c} \PY{o}{=} \PY{l+m+mf}{0.5}
\PY{n}{m} \PY{o}{=} \PY{l+m+mi}{10}
\PY{c+c1}{\PYZsh{} Intervalo da solução}
\PY{n}{t} \PY{o}{=} \PY{n}{np}\PY{o}{.}\PY{n}{linspace}\PY{p}{(}\PY{l+m+mi}{0}\PY{p}{,} \PY{l+m+mi}{5}\PY{p}{,} \PY{l+m+mi}{100}\PY{p}{,} \PY{k+kc}{True}\PY{p}{)}
\end{Verbatim}
\end{tcolorbox}

    \hypertarget{resolvendo-as-edos-numericamente-com-o-muxe9todo-lsoda}{%
\paragraph{Resolvendo as EDOs numericamente com o método
LSODA:}\label{resolvendo-as-edos-numericamente-com-o-muxe9todo-lsoda}}

    \begin{tcolorbox}[breakable, size=fbox, boxrule=1pt, pad at break*=1mm,colback=cellbackground, colframe=cellborder]
\prompt{In}{incolor}{5}{\boxspacing}
\begin{Verbatim}[commandchars=\\\{\}]
\PY{n}{sol} \PY{o}{=} \PY{n}{solve\PYZus{}ivp}\PY{p}{(}\PY{n}{fun}\PY{o}{=}\PY{n}{r}\PY{p}{,} \PY{n}{t\PYZus{}span}\PY{o}{=}\PY{p}{[}\PY{n}{t}\PY{p}{[}\PY{l+m+mi}{0}\PY{p}{]}\PY{p}{,} \PY{n}{t}\PY{p}{[}\PY{o}{\PYZhy{}}\PY{l+m+mi}{1}\PY{p}{]}\PY{p}{]}\PY{p}{,} \PY{n}{y0}\PY{o}{=}\PY{p}{[}\PY{n}{x0}\PY{p}{,} \PY{n}{z0}\PY{p}{,} \PY{n}{vx0}\PY{p}{,} \PY{n}{vz0}\PY{p}{]}\PY{p}{,} \PY{n}{method}\PY{o}{=}\PY{l+s+s2}{\PYZdq{}}\PY{l+s+s2}{LSODA}\PY{l+s+s2}{\PYZdq{}}\PY{p}{,} \PY{n}{t\PYZus{}eval}\PY{o}{=}\PY{n}{t}\PY{p}{,} \PY{n}{args}\PY{o}{=}\PY{p}{(}\PY{n}{g}\PY{p}{,} \PY{n}{c}\PY{p}{,} \PY{n}{m}\PY{p}{)}\PY{p}{)}
\end{Verbatim}
\end{tcolorbox}

    \hypertarget{plotando-gruxe1fico-do-comportamentos-horizontais-em-funuxe7uxe3o-do-tempo}{%
\paragraph{Plotando gráfico do comportamentos horizontais em função do
tempo:}\label{plotando-gruxe1fico-do-comportamentos-horizontais-em-funuxe7uxe3o-do-tempo}}

    \begin{tcolorbox}[breakable, size=fbox, boxrule=1pt, pad at break*=1mm,colback=cellbackground, colframe=cellborder]
\prompt{In}{incolor}{6}{\boxspacing}
\begin{Verbatim}[commandchars=\\\{\}]
\PY{n}{fig}\PY{p}{,} \PY{p}{(}\PY{p}{(}\PY{n}{x}\PY{p}{)}\PY{p}{,} \PY{p}{(}\PY{n}{vx}\PY{p}{)}\PY{p}{)} \PY{o}{=} \PY{n}{plt}\PY{o}{.}\PY{n}{subplots}\PY{p}{(}\PY{l+m+mi}{2}\PY{p}{,} \PY{l+m+mi}{1}\PY{p}{,} \PY{n}{sharex}\PY{o}{=}\PY{k+kc}{True}\PY{p}{)}
\PY{n}{x}\PY{o}{.}\PY{n}{plot}\PY{p}{(}\PY{n}{sol}\PY{o}{.}\PY{n}{t}\PY{p}{,} \PY{n}{sol}\PY{o}{.}\PY{n}{y}\PY{p}{[}\PY{l+m+mi}{0}\PY{p}{]}\PY{p}{,} \PY{n}{c}\PY{o}{=}\PY{l+s+s2}{\PYZdq{}}\PY{l+s+s2}{b}\PY{l+s+s2}{\PYZdq{}}\PY{p}{)}
\PY{n}{x}\PY{o}{.}\PY{n}{set\PYZus{}ylabel}\PY{p}{(}\PY{l+s+s2}{\PYZdq{}}\PY{l+s+s2}{Posição (m)}\PY{l+s+s2}{\PYZdq{}}\PY{p}{)}
\PY{n}{x}\PY{o}{.}\PY{n}{set\PYZus{}title}\PY{p}{(}\PY{l+s+s2}{\PYZdq{}}\PY{l+s+s2}{Posição horizontal em função do tempo}\PY{l+s+s2}{\PYZdq{}}\PY{p}{)}
\PY{n}{x}\PY{o}{.}\PY{n}{set\PYZus{}ylim}\PY{p}{(}\PY{n}{sol}\PY{o}{.}\PY{n}{y}\PY{p}{[}\PY{l+m+mi}{0}\PY{p}{]}\PY{o}{.}\PY{n}{min}\PY{p}{(}\PY{p}{)}\PY{o}{\PYZhy{}}\PY{l+m+mi}{1}\PY{p}{,} \PY{n}{sol}\PY{o}{.}\PY{n}{y}\PY{p}{[}\PY{l+m+mi}{0}\PY{p}{]}\PY{o}{.}\PY{n}{max}\PY{p}{(}\PY{p}{)}\PY{o}{+}\PY{l+m+mi}{1}\PY{p}{)}

\PY{n}{vx}\PY{o}{.}\PY{n}{plot}\PY{p}{(}\PY{n}{sol}\PY{o}{.}\PY{n}{t}\PY{p}{,} \PY{n}{sol}\PY{o}{.}\PY{n}{y}\PY{p}{[}\PY{l+m+mi}{2}\PY{p}{]}\PY{p}{,} \PY{n}{c}\PY{o}{=}\PY{l+s+s2}{\PYZdq{}}\PY{l+s+s2}{orange}\PY{l+s+s2}{\PYZdq{}}\PY{p}{)}
\PY{n}{vx}\PY{o}{.}\PY{n}{set\PYZus{}xlabel}\PY{p}{(}\PY{l+s+s2}{\PYZdq{}}\PY{l+s+s2}{Tempo (s)}\PY{l+s+s2}{\PYZdq{}}\PY{p}{)}
\PY{n}{vx}\PY{o}{.}\PY{n}{set\PYZus{}ylabel}\PY{p}{(}\PY{l+s+sa}{r}\PY{l+s+s2}{\PYZdq{}}\PY{l+s+s2}{Velocidade (\PYZdl{}}\PY{l+s+s2}{\PYZbs{}}\PY{l+s+s2}{frac}\PY{l+s+si}{\PYZob{}m\PYZcb{}}\PY{l+s+si}{\PYZob{}s\PYZcb{}}\PY{l+s+s2}{\PYZdl{})}\PY{l+s+s2}{\PYZdq{}}\PY{p}{)}
\PY{n}{vx}\PY{o}{.}\PY{n}{set\PYZus{}title}\PY{p}{(}\PY{l+s+s2}{\PYZdq{}}\PY{l+s+s2}{Velocidade horizontal em função do tempo}\PY{l+s+s2}{\PYZdq{}}\PY{p}{)}
\PY{n}{vx}\PY{o}{.}\PY{n}{set\PYZus{}xlim}\PY{p}{(}\PY{n}{t}\PY{p}{[}\PY{l+m+mi}{0}\PY{p}{]}\PY{o}{\PYZhy{}}\PY{o}{.}\PY{l+m+mi}{1}\PY{p}{,} \PY{n}{t}\PY{p}{[}\PY{o}{\PYZhy{}}\PY{l+m+mi}{1}\PY{p}{]}\PY{o}{+}\PY{o}{.}\PY{l+m+mi}{1}\PY{p}{)}
\PY{n}{vx}\PY{o}{.}\PY{n}{set\PYZus{}ylim}\PY{p}{(}\PY{n}{sol}\PY{o}{.}\PY{n}{y}\PY{p}{[}\PY{l+m+mi}{2}\PY{p}{]}\PY{o}{.}\PY{n}{min}\PY{p}{(}\PY{p}{)}\PY{o}{\PYZhy{}}\PY{l+m+mi}{1}\PY{p}{,} \PY{n}{sol}\PY{o}{.}\PY{n}{y}\PY{p}{[}\PY{l+m+mi}{2}\PY{p}{]}\PY{o}{.}\PY{n}{max}\PY{p}{(}\PY{p}{)}\PY{o}{+}\PY{l+m+mi}{1}\PY{p}{)}
\end{Verbatim}
\end{tcolorbox}

            \begin{tcolorbox}[breakable, size=fbox, boxrule=.5pt, pad at break*=1mm, opacityfill=0]
\prompt{Out}{outcolor}{6}{\boxspacing}
\begin{Verbatim}[commandchars=\\\{\}]
(-0.9404243892796738, 22.213203435596427)
\end{Verbatim}
\end{tcolorbox}
        
    \begin{center}
    \adjustimage{max size={0.9\linewidth}{0.9\paperheight}}{Relatorio_files/Relatorio_11_1.png}
    \end{center}
    { \hspace*{\fill} \\}
    
    Gráfico da velocidade em função do tempo: note que esta favorável com o
que esperavamos com uma velocidade tendendo a zero, pois, a resistência
freará o prójetil até que pare. Desta forma, a posição estagna
convergindo a um ponto.

\hypertarget{plotando-gruxe1fico-do-comportamento-vertical-em-funuxe7uxe3o-do-tempo}{%
\paragraph{Plotando gráfico do comportamento vertical em função do
tempo:}\label{plotando-gruxe1fico-do-comportamento-vertical-em-funuxe7uxe3o-do-tempo}}

    \begin{tcolorbox}[breakable, size=fbox, boxrule=1pt, pad at break*=1mm,colback=cellbackground, colframe=cellborder]
\prompt{In}{incolor}{7}{\boxspacing}
\begin{Verbatim}[commandchars=\\\{\}]
\PY{n}{fig}\PY{p}{,} \PY{p}{(}\PY{p}{(}\PY{n}{z}\PY{p}{)}\PY{p}{,} \PY{p}{(}\PY{n}{vz}\PY{p}{)}\PY{p}{)} \PY{o}{=} \PY{n}{plt}\PY{o}{.}\PY{n}{subplots}\PY{p}{(}\PY{l+m+mi}{2}\PY{p}{,} \PY{l+m+mi}{1}\PY{p}{,} \PY{n}{sharex}\PY{o}{=}\PY{k+kc}{True}\PY{p}{)}
\PY{n}{z}\PY{o}{.}\PY{n}{plot}\PY{p}{(}\PY{n}{sol}\PY{o}{.}\PY{n}{t}\PY{p}{,} \PY{n}{sol}\PY{o}{.}\PY{n}{y}\PY{p}{[}\PY{l+m+mi}{1}\PY{p}{]}\PY{p}{,} \PY{n}{c}\PY{o}{=}\PY{l+s+s2}{\PYZdq{}}\PY{l+s+s2}{b}\PY{l+s+s2}{\PYZdq{}}\PY{p}{)}
\PY{n}{z}\PY{o}{.}\PY{n}{set\PYZus{}ylabel}\PY{p}{(}\PY{l+s+s2}{\PYZdq{}}\PY{l+s+s2}{Posição (m)}\PY{l+s+s2}{\PYZdq{}}\PY{p}{)}
\PY{n}{z}\PY{o}{.}\PY{n}{set\PYZus{}title}\PY{p}{(}\PY{l+s+s2}{\PYZdq{}}\PY{l+s+s2}{Posição vertical em função do tempo}\PY{l+s+s2}{\PYZdq{}}\PY{p}{)}
\PY{n}{z}\PY{o}{.}\PY{n}{set\PYZus{}ylim}\PY{p}{(}\PY{n}{sol}\PY{o}{.}\PY{n}{y}\PY{p}{[}\PY{l+m+mi}{1}\PY{p}{]}\PY{o}{.}\PY{n}{min}\PY{p}{(}\PY{p}{)}\PY{o}{\PYZhy{}}\PY{l+m+mi}{1}\PY{p}{,} \PY{n}{sol}\PY{o}{.}\PY{n}{y}\PY{p}{[}\PY{l+m+mi}{1}\PY{p}{]}\PY{o}{.}\PY{n}{max}\PY{p}{(}\PY{p}{)}\PY{o}{+}\PY{l+m+mi}{1}\PY{p}{)}

\PY{n}{vz}\PY{o}{.}\PY{n}{plot}\PY{p}{(}\PY{n}{sol}\PY{o}{.}\PY{n}{t}\PY{p}{,} \PY{n}{sol}\PY{o}{.}\PY{n}{y}\PY{p}{[}\PY{l+m+mi}{3}\PY{p}{]}\PY{p}{,} \PY{n}{c}\PY{o}{=}\PY{l+s+s2}{\PYZdq{}}\PY{l+s+s2}{orange}\PY{l+s+s2}{\PYZdq{}}\PY{p}{)}
\PY{n}{vz}\PY{o}{.}\PY{n}{set\PYZus{}ylabel}\PY{p}{(}\PY{l+s+sa}{r}\PY{l+s+s2}{\PYZdq{}}\PY{l+s+s2}{Velocidade (\PYZdl{}}\PY{l+s+s2}{\PYZbs{}}\PY{l+s+s2}{frac}\PY{l+s+si}{\PYZob{}m\PYZcb{}}\PY{l+s+si}{\PYZob{}s\PYZcb{}}\PY{l+s+s2}{\PYZdl{})}\PY{l+s+s2}{\PYZdq{}}\PY{p}{)}
\PY{n}{vz}\PY{o}{.}\PY{n}{set\PYZus{}title}\PY{p}{(}\PY{l+s+s2}{\PYZdq{}}\PY{l+s+s2}{Velocidade vertical em função do tempo}\PY{l+s+s2}{\PYZdq{}}\PY{p}{)}
\PY{n}{vz}\PY{o}{.}\PY{n}{set\PYZus{}xlabel}\PY{p}{(}\PY{l+s+s2}{\PYZdq{}}\PY{l+s+s2}{Tempo (t)}\PY{l+s+s2}{\PYZdq{}}\PY{p}{)}
\PY{n}{vz}\PY{o}{.}\PY{n}{set\PYZus{}xlim}\PY{p}{(}\PY{n}{t}\PY{p}{[}\PY{l+m+mi}{0}\PY{p}{]}\PY{o}{\PYZhy{}}\PY{o}{.}\PY{l+m+mi}{1}\PY{p}{,} \PY{n}{t}\PY{p}{[}\PY{o}{\PYZhy{}}\PY{l+m+mi}{1}\PY{p}{]}\PY{o}{+}\PY{o}{.}\PY{l+m+mi}{1}\PY{p}{)}
\PY{n}{vz}\PY{o}{.}\PY{n}{set\PYZus{}ylim}\PY{p}{(}\PY{n}{sol}\PY{o}{.}\PY{n}{y}\PY{p}{[}\PY{l+m+mi}{3}\PY{p}{]}\PY{o}{.}\PY{n}{min}\PY{p}{(}\PY{p}{)}\PY{o}{\PYZhy{}}\PY{l+m+mi}{1}\PY{p}{,} \PY{n}{sol}\PY{o}{.}\PY{n}{y}\PY{p}{[}\PY{l+m+mi}{3}\PY{p}{]}\PY{o}{.}\PY{n}{max}\PY{p}{(}\PY{p}{)}\PY{o}{+}\PY{l+m+mi}{1}\PY{p}{)}
\end{Verbatim}
\end{tcolorbox}

            \begin{tcolorbox}[breakable, size=fbox, boxrule=.5pt, pad at break*=1mm, opacityfill=0]
\prompt{Out}{outcolor}{7}{\boxspacing}
\begin{Verbatim}[commandchars=\\\{\}]
(-8.349108400192053, 22.213203435596423)
\end{Verbatim}
\end{tcolorbox}
        
    \begin{center}
    \adjustimage{max size={0.9\linewidth}{0.9\paperheight}}{Relatorio_files/Relatorio_13_1.png}
    \end{center}
    { \hspace*{\fill} \\}
    
    Perceba que chegamos no que esperamos novamente. Pois, diferente de
\(x\), \(z\) parte de zero e chega novamente em \(0\), pelo fato que há
uma aceleração descendente. Porém, ao ganhar velocidade, a resistência
faz com que convirja, pois, a resistência é proporcional a velocidade.

\hypertarget{alternativa-b}{%
\subsection{Alternativa (b)}\label{alternativa-b}}

Reaproveitamos as constantes e a função da alternativa anterior.

Dessa foram, para realizar várias soluções diferentes usaremos um
\emph{loop} onde trocaremos somente os ângulos \(\theta\) em um
intervalo de \([0; 90]\) graus.

\hypertarget{loop-com-a-soluuxe7uxe3o-e-plotagem-do-gruxe1fico-por-theta}{%
\paragraph{\texorpdfstring{\emph{Loop} com a solução e plotagem do
gráfico por
\(\theta\):}{Loop com a solução e plotagem do gráfico por \textbackslash{}theta:}}\label{loop-com-a-soluuxe7uxe3o-e-plotagem-do-gruxe1fico-por-theta}}

    \begin{tcolorbox}[breakable, size=fbox, boxrule=1pt, pad at break*=1mm,colback=cellbackground, colframe=cellborder]
\prompt{In}{incolor}{8}{\boxspacing}
\begin{Verbatim}[commandchars=\\\{\}]
\PY{k}{for} \PY{n}{\PYZus{}} \PY{o+ow}{in} \PY{n}{np}\PY{o}{.}\PY{n}{arange}\PY{p}{(}\PY{l+m+mi}{0}\PY{p}{,} \PY{l+m+mi}{91}\PY{p}{,} \PY{l+m+mi}{3}\PY{p}{)}\PY{p}{:}  \PY{c+c1}{\PYZsh{} loop variando \PYZdl{}\PYZbs{}theta\PYZdl{}}
    \PY{c+c1}{\PYZsh{} Tranformando graus em radianos}
    \PY{n}{theta\PYZus{}i} \PY{o}{=} \PY{n}{np}\PY{o}{.}\PY{n}{radians}\PY{p}{(}\PY{n}{\PYZus{}}\PY{p}{)}
    \PY{c+c1}{\PYZsh{} Condições iniciais ara cada angulação}
    \PY{n}{vx0i} \PY{o}{=} \PY{n}{v0}\PY{o}{*}\PY{n}{np}\PY{o}{.}\PY{n}{cos}\PY{p}{(}\PY{n}{theta\PYZus{}i}\PY{p}{)}
    \PY{n}{vz0i} \PY{o}{=} \PY{n}{v0}\PY{o}{*}\PY{n}{np}\PY{o}{.}\PY{n}{sin}\PY{p}{(}\PY{n}{theta\PYZus{}i}\PY{p}{)}
    \PY{c+c1}{\PYZsh{} Solução para cada eixo}
    \PY{n}{sol\PYZus{}i} \PY{o}{=} \PY{n}{solve\PYZus{}ivp}\PY{p}{(}\PY{n}{r}\PY{p}{,} \PY{p}{[}\PY{n}{t}\PY{p}{[}\PY{l+m+mi}{0}\PY{p}{]}\PY{p}{,} \PY{n}{t}\PY{p}{[}\PY{o}{\PYZhy{}}\PY{l+m+mi}{1}\PY{p}{]}\PY{p}{]}\PY{p}{,} \PY{p}{[}\PY{n}{x0}\PY{p}{,} \PY{n}{z0}\PY{p}{,} \PY{n}{vx0i}\PY{p}{,} \PY{n}{vz0i}\PY{p}{]}\PY{p}{,} \PY{l+s+s2}{\PYZdq{}}\PY{l+s+s2}{LSODA}\PY{l+s+s2}{\PYZdq{}}\PY{p}{,} \PY{n}{t}\PY{p}{,} \PY{n}{args}\PY{o}{=}\PY{p}{(}\PY{n}{g}\PY{p}{,} \PY{n}{c}\PY{p}{,} \PY{n}{m}\PY{p}{)}\PY{p}{)}
    \PY{c+c1}{\PYZsh{} Plotagem do grafico para cada theta}
    \PY{n}{plt}\PY{o}{.}\PY{n}{plot}\PY{p}{(}\PY{n}{sol\PYZus{}i}\PY{o}{.}\PY{n}{y}\PY{p}{[}\PY{l+m+mi}{0}\PY{p}{]}\PY{p}{,} \PY{n}{sol\PYZus{}i}\PY{o}{.}\PY{n}{y}\PY{p}{[}\PY{l+m+mi}{1}\PY{p}{]}\PY{p}{,} \PY{n}{c}\PY{o}{=}\PY{n}{cmap}\PY{o}{.}\PY{n}{to\PYZus{}rgba}\PY{p}{(}\PY{n}{\PYZus{}} \PY{o}{+} \PY{l+m+mi}{1}\PY{p}{)}\PY{p}{)}


\PY{n}{plt}\PY{o}{.}\PY{n}{colorbar}\PY{p}{(}\PY{n}{cmap}\PY{p}{,} \PY{n}{ticks}\PY{o}{=}\PY{n}{np}\PY{o}{.}\PY{n}{arange}\PY{p}{(}\PY{l+m+mi}{0}\PY{p}{,} \PY{l+m+mi}{95}\PY{p}{,} \PY{l+m+mi}{10}\PY{p}{)}\PY{p}{,} \PY{n}{label}\PY{o}{=}\PY{l+s+sa}{r}\PY{l+s+s2}{\PYZdq{}}\PY{l+s+s2}{ângulos (\PYZdl{}}\PY{l+s+s2}{\PYZbs{}}\PY{l+s+s2}{theta\PYZdl{}°)}\PY{l+s+s2}{\PYZdq{}}\PY{p}{)}
\PY{n}{plt}\PY{o}{.}\PY{n}{xlim}\PY{p}{(}\PY{o}{\PYZhy{}}\PY{o}{.}\PY{l+m+mi}{2}\PY{p}{,} \PY{l+m+mi}{32}\PY{p}{)}
\PY{n}{plt}\PY{o}{.}\PY{n}{ylim}\PY{p}{(}\PY{l+m+mi}{0}\PY{p}{,} \PY{l+m+mi}{24}\PY{p}{)}
\PY{n}{plt}\PY{o}{.}\PY{n}{title}\PY{p}{(}\PY{l+s+sa}{r}\PY{l+s+s2}{\PYZdq{}}\PY{l+s+s2}{Posição para diferentes valores de \PYZdl{}}\PY{l+s+s2}{\PYZbs{}}\PY{l+s+s2}{theta\PYZdl{}}\PY{l+s+s2}{\PYZdq{}}\PY{p}{)}
\PY{n}{plt}\PY{o}{.}\PY{n}{xlabel}\PY{p}{(}\PY{l+s+s2}{\PYZdq{}}\PY{l+s+s2}{Horizontal (m)}\PY{l+s+s2}{\PYZdq{}}\PY{p}{)}
\PY{n}{plt}\PY{o}{.}\PY{n}{ylabel}\PY{p}{(}\PY{l+s+s2}{\PYZdq{}}\PY{l+s+s2}{Vertical (m)}\PY{l+s+s2}{\PYZdq{}}\PY{p}{)}
\end{Verbatim}
\end{tcolorbox}

            \begin{tcolorbox}[breakable, size=fbox, boxrule=.5pt, pad at break*=1mm, opacityfill=0]
\prompt{Out}{outcolor}{8}{\boxspacing}
\begin{Verbatim}[commandchars=\\\{\}]
Text(0, 0.5, 'Vertical (m)')
\end{Verbatim}
\end{tcolorbox}
        
    \begin{center}
    \adjustimage{max size={0.9\linewidth}{0.9\paperheight}}{Relatorio_files/Relatorio_15_1.png}
    \end{center}
    { \hspace*{\fill} \\}
    
    Note que, o alcance máximo fica próximo dos 25° graus, menor que os 45°
graus no caso sem resistência do ar. Pois, quanto mais tempo durar o
percurso mais maior será desaceleração pela resistência do ar. Além
disso, para ele se manter mais tempo no ar ele deve se angular mais,
logo, perdendo velocidade horizontal. Dessa forma, quanto maior for o
fator da resistência do ar, menor será o angulo para o alcance máximo.

\hypertarget{alternativa-c}{%
\subsubsection{Alternativa (c)}\label{alternativa-c}}

Para agrupar estes pontos usaremos o parâmetro \emph{event} da função
\emph{solve\_ivp}. Esta função encontrará os demais valores, onde a
variável de nosso interesse é zero. Dessa forma, pegaremos os pontos
onde \(\frac{dz}{dt}=0\), ou seja, o poto onde \(z\) é máximo e \(z=0\)
onde o projetil está no chão, logo, podemos adquirir o máximo alcance.

\hypertarget{definindo-as-funuxe7uxf5es-onde-queremos-capturar-os-pontos}{%
\paragraph{Definindo as funções onde queremos capturar os
pontos:}\label{definindo-as-funuxe7uxf5es-onde-queremos-capturar-os-pontos}}

    \begin{tcolorbox}[breakable, size=fbox, boxrule=1pt, pad at break*=1mm,colback=cellbackground, colframe=cellborder]
\prompt{In}{incolor}{9}{\boxspacing}
\begin{Verbatim}[commandchars=\\\{\}]
\PY{k}{def} \PY{n+nf}{topo}\PY{p}{(}\PY{n}{t}\PY{p}{,} \PY{n}{r}\PY{p}{,} \PY{n}{g}\PY{p}{,} \PY{n}{c}\PY{p}{,} \PY{n}{m}\PY{p}{)}\PY{p}{:}  \PY{c+c1}{\PYZsh{} Função para capturar pontos, no qual a velocidade em z é zero}
    \PY{n}{x}\PY{p}{,} \PY{n}{z}\PY{p}{,} \PY{n}{vx}\PY{p}{,} \PY{n}{vz} \PY{o}{=} \PY{n}{r}
    \PY{k}{return} \PY{n}{vz}


\PY{k}{def} \PY{n+nf}{alcance}\PY{p}{(}\PY{n}{t}\PY{p}{,} \PY{n}{r}\PY{p}{,} \PY{n}{g}\PY{p}{,} \PY{n}{c}\PY{p}{,} \PY{n}{m}\PY{p}{)}\PY{p}{:}  \PY{c+c1}{\PYZsh{} Função para capturar pontos, no qual o valor z é zero}
    \PY{n}{x}\PY{p}{,} \PY{n}{z}\PY{p}{,} \PY{n}{vx}\PY{p}{,} \PY{n}{vz} \PY{o}{=} \PY{n}{r}
    \PY{k}{return} \PY{n}{z}
\end{Verbatim}
\end{tcolorbox}

    \hypertarget{fazendo-soluuxe7uxf5es-para-varios-theta-igual-o-que-foi-feito-no-grafico-anterio-porem-com-mais-pontos}{%
\paragraph{\texorpdfstring{Fazendo soluções para varios \(\theta\) igual
o que foi feito no grafico anterio, porem com mais
pontos:}{Fazendo soluções para varios \textbackslash{}theta igual o que foi feito no grafico anterio, porem com mais pontos:}}\label{fazendo-soluuxe7uxf5es-para-varios-theta-igual-o-que-foi-feito-no-grafico-anterio-porem-com-mais-pontos}}

    \begin{tcolorbox}[breakable, size=fbox, boxrule=1pt, pad at break*=1mm,colback=cellbackground, colframe=cellborder]
\prompt{In}{incolor}{10}{\boxspacing}
\begin{Verbatim}[commandchars=\\\{\}]
\PY{n}{pontos} \PY{o}{=} \PY{n}{np}\PY{o}{.}\PY{n}{array}\PY{p}{(}\PY{p}{[}\PY{p}{[}\PY{p}{]}\PY{p}{,} \PY{p}{[}\PY{p}{]}\PY{p}{,} \PY{p}{[}\PY{p}{]}\PY{p}{]}\PY{p}{)}  \PY{c+c1}{\PYZsh{} Criando array para recolher os pontos}

\PY{k}{for} \PY{n}{\PYZus{}} \PY{o+ow}{in} \PY{n}{np}\PY{o}{.}\PY{n}{linspace}\PY{p}{(}\PY{l+m+mi}{0}\PY{p}{,} \PY{l+m+mi}{90}\PY{p}{,} \PY{l+m+mi}{200}\PY{p}{,} \PY{k+kc}{True}\PY{p}{)}\PY{p}{:}  \PY{c+c1}{\PYZsh{} loop variando \PYZdl{}\PYZbs{}theta\PYZdl{}}
    \PY{c+c1}{\PYZsh{} Tranformando graus em radianos}
    \PY{n}{theta\PYZus{}i} \PY{o}{=} \PY{n}{np}\PY{o}{.}\PY{n}{radians}\PY{p}{(}\PY{n}{\PYZus{}}\PY{p}{)}
    \PY{c+c1}{\PYZsh{} Condições iniciais para cada angulação}
    \PY{n}{vz0i} \PY{o}{=} \PY{n}{v0}\PY{o}{*}\PY{n}{np}\PY{o}{.}\PY{n}{sin}\PY{p}{(}\PY{n}{theta\PYZus{}i}\PY{p}{)}
    \PY{n}{vx0i} \PY{o}{=} \PY{n}{v0}\PY{o}{*}\PY{n}{np}\PY{o}{.}\PY{n}{cos}\PY{p}{(}\PY{n}{theta\PYZus{}i}\PY{p}{)}
    \PY{c+c1}{\PYZsh{} Solução do eixos x e z}
    \PY{n}{sol\PYZus{}i} \PY{o}{=} \PY{n}{solve\PYZus{}ivp}\PY{p}{(}\PY{n}{fun}\PY{o}{=}\PY{n}{r}\PY{p}{,} \PY{n}{t\PYZus{}span}\PY{o}{=}\PY{p}{[}\PY{n}{t}\PY{p}{[}\PY{l+m+mi}{0}\PY{p}{]}\PY{p}{,} \PY{n}{t}\PY{p}{[}\PY{o}{\PYZhy{}}\PY{l+m+mi}{1}\PY{p}{]}\PY{p}{]}\PY{p}{,} \PY{n}{y0}\PY{o}{=}\PY{p}{[}\PY{n}{x0}\PY{p}{,} \PY{n}{z0}\PY{p}{,} \PY{n}{vx0i}\PY{p}{,} \PY{n}{vz0i}\PY{p}{]}\PY{p}{,} \PY{n}{method}\PY{o}{=}\PY{l+s+s2}{\PYZdq{}}\PY{l+s+s2}{LSODA}\PY{l+s+s2}{\PYZdq{}}\PY{p}{,} \PY{n}{t\PYZus{}eval}\PY{o}{=}\PY{n}{t}\PY{p}{,} \PY{n}{args}\PY{o}{=}\PY{p}{(}\PY{n}{g}\PY{p}{,} \PY{n}{c}\PY{p}{,} \PY{n}{m}\PY{p}{)}\PY{p}{,} \PY{n}{events}\PY{o}{=}\PY{p}{(}\PY{n}{topo}\PY{p}{,} \PY{n}{alcance}\PY{p}{)}\PY{p}{,} \PY{n}{dense\PYZus{}output}\PY{o}{=}\PY{k+kc}{True}\PY{p}{)}
    \PY{c+c1}{\PYZsh{} Recolhendo somente os pontos de interesse gerados pela soção}
    \PY{n}{pontos} \PY{o}{=} \PY{n}{np}\PY{o}{.}\PY{n}{append}\PY{p}{(}\PY{n}{pontos}\PY{p}{,} \PY{p}{[}\PY{p}{[}\PY{n}{\PYZus{}}\PY{p}{]}\PY{p}{,} \PY{n}{sol\PYZus{}i}\PY{o}{.}\PY{n}{y\PYZus{}events}\PY{p}{[}\PY{l+m+mi}{0}\PY{p}{]}\PY{p}{[}\PY{p}{:}\PY{p}{,} \PY{l+m+mi}{1}\PY{p}{]}\PY{p}{,} \PY{p}{[}
        \PY{n}{sol\PYZus{}i}\PY{o}{.}\PY{n}{y\PYZus{}events}\PY{p}{[}\PY{l+m+mi}{1}\PY{p}{]}\PY{p}{[}\PY{p}{:}\PY{p}{,} \PY{l+m+mi}{0}\PY{p}{]}\PY{p}{[}\PY{l+m+mi}{1}\PY{p}{]} \PY{k}{if} \PY{n}{\PYZus{}} \PY{o+ow}{not} \PY{o+ow}{in} \PY{p}{[}\PY{l+m+mi}{0}\PY{p}{,} \PY{l+m+mi}{90}\PY{p}{]} \PY{k}{else} \PY{n}{sol\PYZus{}i}\PY{o}{.}\PY{n}{y\PYZus{}events}\PY{p}{[}\PY{l+m+mi}{1}\PY{p}{]}\PY{p}{[}\PY{p}{:}\PY{p}{,} \PY{l+m+mi}{0}\PY{p}{]}\PY{p}{[}\PY{l+m+mi}{0}\PY{p}{]}\PY{p}{]}\PY{p}{]}\PY{p}{,} \PY{l+m+mi}{1}\PY{p}{)}
    \PY{c+c1}{\PYZsh{} note que foi feito uma condição para as situações, onde há somente uma raiz para z = 0}
\end{Verbatim}
\end{tcolorbox}

    \hypertarget{plotando-grafico-da-altura-e-alcance-maximo-em-funuxe7uxe3o-do-angulo}{%
\paragraph{Plotando grafico da altura e alcance maximo em função do
angulo:}\label{plotando-grafico-da-altura-e-alcance-maximo-em-funuxe7uxe3o-do-angulo}}

    \begin{tcolorbox}[breakable, size=fbox, boxrule=1pt, pad at break*=1mm,colback=cellbackground, colframe=cellborder]
\prompt{In}{incolor}{11}{\boxspacing}
\begin{Verbatim}[commandchars=\\\{\}]
\PY{n}{fig}\PY{p}{,} \PY{p}{(}\PY{p}{(}\PY{n}{ax1}\PY{p}{)}\PY{p}{,} \PY{p}{(}\PY{n}{ax2}\PY{p}{)}\PY{p}{)} \PY{o}{=} \PY{n}{plt}\PY{o}{.}\PY{n}{subplots}\PY{p}{(}\PY{l+m+mi}{2}\PY{p}{,} \PY{l+m+mi}{1}\PY{p}{,} \PY{n}{sharex}\PY{o}{=}\PY{k+kc}{True}\PY{p}{)}

\PY{n}{ax1}\PY{o}{.}\PY{n}{plot}\PY{p}{(}\PY{n}{pontos}\PY{p}{[}\PY{l+m+mi}{0}\PY{p}{]}\PY{p}{,} \PY{n}{pontos}\PY{p}{[}\PY{l+m+mi}{1}\PY{p}{]}\PY{p}{,} \PY{n}{c}\PY{o}{=}\PY{l+s+s2}{\PYZdq{}}\PY{l+s+s2}{r}\PY{l+s+s2}{\PYZdq{}}\PY{p}{)}
\PY{n}{ax1}\PY{o}{.}\PY{n}{set\PYZus{}title}\PY{p}{(}\PY{l+s+s2}{\PYZdq{}}\PY{l+s+s2}{Altura máxima em função do angulo}\PY{l+s+s2}{\PYZdq{}}\PY{p}{)}
\PY{n}{ax1}\PY{o}{.}\PY{n}{set\PYZus{}ylim}\PY{p}{(}\PY{n}{pontos}\PY{p}{[}\PY{l+m+mi}{1}\PY{p}{]}\PY{o}{.}\PY{n}{min}\PY{p}{(}\PY{p}{)}\PY{o}{\PYZhy{}}\PY{o}{.}\PY{l+m+mi}{5}\PY{p}{,} \PY{n}{pontos}\PY{p}{[}\PY{l+m+mi}{1}\PY{p}{]}\PY{o}{.}\PY{n}{max}\PY{p}{(}\PY{p}{)}\PY{o}{+}\PY{o}{.}\PY{l+m+mi}{5}\PY{p}{)}
\PY{n}{ax1}\PY{o}{.}\PY{n}{set\PYZus{}ylabel}\PY{p}{(}\PY{l+s+s2}{\PYZdq{}}\PY{l+s+s2}{Altura máxima (m)}\PY{l+s+s2}{\PYZdq{}}\PY{p}{)}

\PY{n}{ax2}\PY{o}{.}\PY{n}{plot}\PY{p}{(}\PY{n}{pontos}\PY{p}{[}\PY{l+m+mi}{0}\PY{p}{]}\PY{p}{,} \PY{n}{pontos}\PY{p}{[}\PY{l+m+mi}{2}\PY{p}{]}\PY{p}{,} \PY{n}{c}\PY{o}{=}\PY{l+s+s2}{\PYZdq{}}\PY{l+s+s2}{b}\PY{l+s+s2}{\PYZdq{}}\PY{p}{)}
\PY{n}{ax2}\PY{o}{.}\PY{n}{set\PYZus{}xlim}\PY{p}{(}\PY{n}{pontos}\PY{p}{[}\PY{l+m+mi}{0}\PY{p}{]}\PY{p}{[}\PY{l+m+mi}{0}\PY{p}{]}\PY{o}{\PYZhy{}}\PY{l+m+mi}{1}\PY{p}{,} \PY{n}{pontos}\PY{p}{[}\PY{l+m+mi}{0}\PY{p}{]}\PY{p}{[}\PY{o}{\PYZhy{}}\PY{l+m+mi}{1}\PY{p}{]}\PY{o}{+}\PY{l+m+mi}{1}\PY{p}{)}
\PY{n}{ax2}\PY{o}{.}\PY{n}{set\PYZus{}title}\PY{p}{(}\PY{l+s+s2}{\PYZdq{}}\PY{l+s+s2}{alcance máxima em função do angulo}\PY{l+s+s2}{\PYZdq{}}\PY{p}{)}
\PY{n}{ax2}\PY{o}{.}\PY{n}{set\PYZus{}ylim}\PY{p}{(}\PY{n}{pontos}\PY{p}{[}\PY{l+m+mi}{2}\PY{p}{]}\PY{o}{.}\PY{n}{min}\PY{p}{(}\PY{p}{)}\PY{o}{\PYZhy{}}\PY{l+m+mi}{1}\PY{p}{,} \PY{n}{pontos}\PY{p}{[}\PY{l+m+mi}{2}\PY{p}{]}\PY{o}{.}\PY{n}{max}\PY{p}{(}\PY{p}{)}\PY{o}{+}\PY{l+m+mi}{1}\PY{p}{)}
\PY{n}{ax2}\PY{o}{.}\PY{n}{set\PYZus{}xlabel}\PY{p}{(}\PY{l+s+s2}{\PYZdq{}}\PY{l+s+s2}{Ângulo (graus)}\PY{l+s+s2}{\PYZdq{}}\PY{p}{)}
\PY{n}{ax2}\PY{o}{.}\PY{n}{set\PYZus{}xticks}\PY{p}{(}\PY{n}{np}\PY{o}{.}\PY{n}{arange}\PY{p}{(}\PY{l+m+mi}{0}\PY{p}{,}\PY{l+m+mi}{91}\PY{p}{,}\PY{l+m+mi}{15}\PY{p}{)}\PY{p}{)}
\PY{n}{ax2}\PY{o}{.}\PY{n}{set\PYZus{}ylabel}\PY{p}{(}\PY{l+s+s2}{\PYZdq{}}\PY{l+s+s2}{Alcance máximo (m)}\PY{l+s+s2}{\PYZdq{}}\PY{p}{)}
\end{Verbatim}
\end{tcolorbox}

            \begin{tcolorbox}[breakable, size=fbox, boxrule=.5pt, pad at break*=1mm, opacityfill=0]
\prompt{Out}{outcolor}{11}{\boxspacing}
\begin{Verbatim}[commandchars=\\\{\}]
Text(0, 0.5, 'Alcance máximo (m)')
\end{Verbatim}
\end{tcolorbox}
        
    \begin{center}
    \adjustimage{max size={0.9\linewidth}{0.9\paperheight}}{Relatorio_files/Relatorio_21_1.png}
    \end{center}
    { \hspace*{\fill} \\}
    
    Observe que, a altura maixma está em 90 graus, pois toda a componente da
velocidade inicial esta em \(z\). Além disto, como vimos no grafico
anterior o alcance maximo esta próximo de 25°.

    \hypertarget{questuxe3o-2}{%
\section{Questão 2}\label{questuxe3o-2}}

Semelhante à questão anterior, porém agora com vento. Ou seja, teremos
mais duas contantes interagindo na equação, uma para o eixo \(x\) e
outra para o \(z\). Desta forma, temos:
\[\vec{v}_{rel} =\vec{v} - \vec{v}_w \] considere
\(\vec{v_w} = u\hat{\imath} + w\hat{k}\).

As condições iniciais são as mesmas.

\hypertarget{alternativa-a}{%
\subsection{Alternativa (a)}\label{alternativa-a}}

Para resolver esta questão precisamos desmembrar o vetor posição em suas
componentes.
\[ m \frac{d^2\vec{r}}{dt^2} =-mg\hat{k} - c\vec{v}_{rel}|\vec{v}_{rel}|\]

Dessa forma, temos duas EDOs de segunda ordem.

\[ \frac{d^2x}{dt^2} = - \frac{c}{m} \left(\frac{dx}{dt}-u\right) \sqrt{\left(\frac{dx}{dt}-u \right)^2 + \left(\frac{dz}{dt} - w \right)^2}
\]
\[ \frac{d^2x}{dt^2} = - \frac{c}{m} \left(\frac{dz}{dt}-w\right) \sqrt{\left(\frac{dx}{dt}-u \right)^2 + \left(\frac{dz}{dt} - w \right)^2}
\]

    \hypertarget{definindo-funuxe7uxe3o-das-edos-para-resolvuxea-las}{%
\paragraph{Definindo função das EDOs para
resolvê-las:}\label{definindo-funuxe7uxe3o-das-edos-para-resolvuxea-las}}

    \begin{tcolorbox}[breakable, size=fbox, boxrule=1pt, pad at break*=1mm,colback=cellbackground, colframe=cellborder]
\prompt{In}{incolor}{12}{\boxspacing}
\begin{Verbatim}[commandchars=\\\{\}]
\PY{k}{def} \PY{n+nf}{r\PYZus{}v}\PY{p}{(}\PY{n}{t}\PY{p}{,} \PY{n}{r}\PY{p}{,} \PY{n}{g}\PY{p}{,} \PY{n}{c}\PY{p}{,} \PY{n}{m}\PY{p}{,} \PY{n}{u}\PY{p}{,} \PY{n}{w}\PY{p}{)}\PY{p}{:}
    \PY{n}{x}\PY{p}{,} \PY{n}{z}\PY{p}{,} \PY{n}{vx}\PY{p}{,} \PY{n}{vz} \PY{o}{=} \PY{n}{r}
    \PY{c+c1}{\PYZsh{} Derivada da segunda de x}
    \PY{n}{ddx} \PY{o}{=} \PY{o}{\PYZhy{}} \PY{n}{c}\PY{o}{/}\PY{n}{m} \PY{o}{*} \PY{p}{(}\PY{n}{vx} \PY{o}{\PYZhy{}} \PY{n}{u}\PY{p}{)} \PY{o}{*} \PY{n}{np}\PY{o}{.}\PY{n}{hypot}\PY{p}{(}\PY{n}{vx} \PY{o}{\PYZhy{}} \PY{n}{u}\PY{p}{,} \PY{n}{vz} \PY{o}{\PYZhy{}} \PY{n}{w}\PY{p}{)}
    \PY{c+c1}{\PYZsh{} Derivada da segunda de z}
    \PY{n}{ddz} \PY{o}{=} \PY{o}{\PYZhy{}} \PY{n}{g} \PY{o}{\PYZhy{}} \PY{n}{c}\PY{o}{/}\PY{n}{m} \PY{o}{*} \PY{p}{(}\PY{n}{vz} \PY{o}{\PYZhy{}} \PY{n}{w}\PY{p}{)} \PY{o}{*} \PY{n}{np}\PY{o}{.}\PY{n}{hypot}\PY{p}{(}\PY{n}{vx} \PY{o}{\PYZhy{}} \PY{n}{u}\PY{p}{,} \PY{n}{vz} \PY{o}{\PYZhy{}} \PY{n}{w}\PY{p}{)}
    \PY{k}{return} \PY{p}{[}\PY{n}{vx}\PY{p}{,} \PY{n}{vz}\PY{p}{,} \PY{n}{ddx}\PY{p}{,} \PY{n}{ddz}\PY{p}{]}
\end{Verbatim}
\end{tcolorbox}

    A função \texttt{np.hypot()} calcula a hipotenusa o que é o mesmo que o
modulo do vetor.

\hypertarget{definindo-condiuxe7uxf5es-iniciais-constantes-e-valores-da-soluuxe7uxe3o}{%
\paragraph{Definindo condições iniciais, constantes e valores da
solução:}\label{definindo-condiuxe7uxf5es-iniciais-constantes-e-valores-da-soluuxe7uxe3o}}

    \begin{tcolorbox}[breakable, size=fbox, boxrule=1pt, pad at break*=1mm,colback=cellbackground, colframe=cellborder]
\prompt{In}{incolor}{13}{\boxspacing}
\begin{Verbatim}[commandchars=\\\{\}]
\PY{c+c1}{\PYZsh{} CondiçÕes iniciais}
\PY{n}{x0} \PY{o}{=} \PY{l+m+mi}{0}
\PY{n}{z0} \PY{o}{=} \PY{l+m+mi}{0}
\PY{n}{v0} \PY{o}{=} \PY{l+m+mi}{30}
\PY{n}{theta} \PY{o}{=} \PY{n}{np}\PY{o}{.}\PY{n}{pi}\PY{o}{/}\PY{l+m+mi}{4}
\PY{n}{vx0} \PY{o}{=} \PY{n}{v0}\PY{o}{*}\PY{n}{np}\PY{o}{.}\PY{n}{cos}\PY{p}{(}\PY{n}{theta}\PY{p}{)}
\PY{n}{vz0} \PY{o}{=} \PY{n}{v0}\PY{o}{*}\PY{n}{np}\PY{o}{.}\PY{n}{sin}\PY{p}{(}\PY{n}{theta}\PY{p}{)}
\PY{c+c1}{\PYZsh{} Constantes}
\PY{n}{g} \PY{o}{=} \PY{l+m+mf}{9.8}
\PY{n}{c} \PY{o}{=} \PY{l+m+mi}{3}
\PY{n}{m} \PY{o}{=} \PY{l+m+mi}{5}
\PY{n}{u} \PY{o}{=} \PY{l+m+mi}{2}
\PY{n}{w} \PY{o}{=} \PY{l+m+mi}{3}
\PY{c+c1}{\PYZsh{} Intervalo da solução}
\PY{n}{t} \PY{o}{=} \PY{n}{np}\PY{o}{.}\PY{n}{linspace}\PY{p}{(}\PY{l+m+mi}{0}\PY{p}{,} \PY{l+m+mi}{8}\PY{p}{,} \PY{l+m+mi}{300}\PY{p}{,} \PY{k+kc}{True}\PY{p}{)}
\end{Verbatim}
\end{tcolorbox}

    \hypertarget{resolvendo-as-edos-numericamente-com-o-muxe9todo-lsoda}{%
\paragraph{Resolvendo as EDOs numericamente com o método
LSODA:}\label{resolvendo-as-edos-numericamente-com-o-muxe9todo-lsoda}}

    \begin{tcolorbox}[breakable, size=fbox, boxrule=1pt, pad at break*=1mm,colback=cellbackground, colframe=cellborder]
\prompt{In}{incolor}{14}{\boxspacing}
\begin{Verbatim}[commandchars=\\\{\}]
\PY{n}{sol\PYZus{}v} \PY{o}{=} \PY{n}{solve\PYZus{}ivp}\PY{p}{(}\PY{n}{fun}\PY{o}{=}\PY{n}{r\PYZus{}v}\PY{p}{,} \PY{n}{t\PYZus{}span}\PY{o}{=}\PY{p}{[}\PY{n}{t}\PY{p}{[}\PY{l+m+mi}{0}\PY{p}{]}\PY{p}{,} \PY{n}{t}\PY{p}{[}\PY{o}{\PYZhy{}}\PY{l+m+mi}{1}\PY{p}{]}\PY{p}{]}\PY{p}{,} \PY{n}{y0}\PY{o}{=}\PY{p}{[}\PY{n}{x0}\PY{p}{,} \PY{n}{z0}\PY{p}{,} \PY{n}{vx0}\PY{p}{,} \PY{n}{vz0}\PY{p}{]}\PY{p}{,} \PY{n}{args}\PY{o}{=}\PY{p}{(}
    \PY{n}{g}\PY{p}{,} \PY{n}{c}\PY{p}{,} \PY{n}{m}\PY{p}{,} \PY{n}{u}\PY{p}{,} \PY{n}{w}\PY{p}{)}\PY{p}{,} \PY{n}{t\PYZus{}eval}\PY{o}{=}\PY{n}{t}\PY{p}{,} \PY{n}{method}\PY{o}{=}\PY{l+s+s2}{\PYZdq{}}\PY{l+s+s2}{LSODA}\PY{l+s+s2}{\PYZdq{}}\PY{p}{)}
\end{Verbatim}
\end{tcolorbox}

    \hypertarget{plotando-gruxe1fico-da-soluuxe7uxe3o-para-velocidade-e-posiuxe7uxe3o-em-funuxe7uxe3o-do-tempo}{%
\paragraph{Plotando gráfico da solução para velocidade e posição em
função do
tempo:}\label{plotando-gruxe1fico-da-soluuxe7uxe3o-para-velocidade-e-posiuxe7uxe3o-em-funuxe7uxe3o-do-tempo}}

    \begin{tcolorbox}[breakable, size=fbox, boxrule=1pt, pad at break*=1mm,colback=cellbackground, colframe=cellborder]
\prompt{In}{incolor}{15}{\boxspacing}
\begin{Verbatim}[commandchars=\\\{\}]
\PY{n}{fig}\PY{p}{,} \PY{p}{(}\PY{p}{(}\PY{n}{xu}\PY{p}{)}\PY{p}{,} \PY{p}{(}\PY{n}{vxu}\PY{p}{)}\PY{p}{)} \PY{o}{=} \PY{n}{plt}\PY{o}{.}\PY{n}{subplots}\PY{p}{(}\PY{l+m+mi}{2}\PY{p}{,} \PY{l+m+mi}{1}\PY{p}{,} \PY{n}{sharex}\PY{o}{=}\PY{k+kc}{True}\PY{p}{)}
\PY{n}{xu}\PY{o}{.}\PY{n}{plot}\PY{p}{(}\PY{n}{sol\PYZus{}v}\PY{o}{.}\PY{n}{t}\PY{p}{,} \PY{n}{sol\PYZus{}v}\PY{o}{.}\PY{n}{y}\PY{p}{[}\PY{l+m+mi}{0}\PY{p}{]}\PY{p}{,} \PY{n}{c}\PY{o}{=}\PY{l+s+s2}{\PYZdq{}}\PY{l+s+s2}{b}\PY{l+s+s2}{\PYZdq{}}\PY{p}{)}
\PY{n}{xu}\PY{o}{.}\PY{n}{set\PYZus{}ylabel}\PY{p}{(}\PY{l+s+s2}{\PYZdq{}}\PY{l+s+s2}{Posição (m)}\PY{l+s+s2}{\PYZdq{}}\PY{p}{)}
\PY{n}{xu}\PY{o}{.}\PY{n}{set\PYZus{}title}\PY{p}{(}\PY{l+s+s2}{\PYZdq{}}\PY{l+s+s2}{Posição horizontal em fução do tempo}\PY{l+s+s2}{\PYZdq{}}\PY{p}{)}
\PY{n}{xu}\PY{o}{.}\PY{n}{set\PYZus{}ylim}\PY{p}{(}\PY{n}{sol\PYZus{}v}\PY{o}{.}\PY{n}{y}\PY{p}{[}\PY{l+m+mi}{0}\PY{p}{]}\PY{o}{.}\PY{n}{min}\PY{p}{(}\PY{p}{)} \PY{o}{\PYZhy{}} \PY{l+m+mi}{1}\PY{p}{,} \PY{n}{sol\PYZus{}v}\PY{o}{.}\PY{n}{y}\PY{p}{[}\PY{l+m+mi}{0}\PY{p}{]}\PY{o}{.}\PY{n}{max}\PY{p}{(}\PY{p}{)}\PY{o}{+}\PY{l+m+mi}{1}\PY{p}{)}

\PY{n}{vxu}\PY{o}{.}\PY{n}{plot}\PY{p}{(}\PY{n}{sol\PYZus{}v}\PY{o}{.}\PY{n}{t}\PY{p}{,} \PY{n}{sol\PYZus{}v}\PY{o}{.}\PY{n}{y}\PY{p}{[}\PY{l+m+mi}{2}\PY{p}{]}\PY{p}{,} \PY{n}{c}\PY{o}{=}\PY{l+s+s2}{\PYZdq{}}\PY{l+s+s2}{orange}\PY{l+s+s2}{\PYZdq{}}\PY{p}{)}
\PY{n}{vxu}\PY{o}{.}\PY{n}{set\PYZus{}xlabel}\PY{p}{(}\PY{l+s+s2}{\PYZdq{}}\PY{l+s+s2}{Tempo (s)}\PY{l+s+s2}{\PYZdq{}}\PY{p}{)}
\PY{n}{vxu}\PY{o}{.}\PY{n}{set\PYZus{}ylabel}\PY{p}{(}\PY{l+s+sa}{r}\PY{l+s+s2}{\PYZdq{}}\PY{l+s+s2}{Velocidade (\PYZdl{}}\PY{l+s+s2}{\PYZbs{}}\PY{l+s+s2}{frac}\PY{l+s+si}{\PYZob{}m\PYZcb{}}\PY{l+s+si}{\PYZob{}s\PYZcb{}}\PY{l+s+s2}{\PYZdl{})}\PY{l+s+s2}{\PYZdq{}}\PY{p}{)}
\PY{n}{vxu}\PY{o}{.}\PY{n}{set\PYZus{}title}\PY{p}{(}\PY{l+s+s2}{\PYZdq{}}\PY{l+s+s2}{Velocidade horizontal em função do tempo}\PY{l+s+s2}{\PYZdq{}}\PY{p}{)}
\PY{n}{vxu}\PY{o}{.}\PY{n}{set\PYZus{}xlim}\PY{p}{(}\PY{n}{t}\PY{p}{[}\PY{l+m+mi}{0}\PY{p}{]}\PY{o}{\PYZhy{}}\PY{o}{.}\PY{l+m+mi}{1}\PY{p}{,} \PY{n}{t}\PY{p}{[}\PY{o}{\PYZhy{}}\PY{l+m+mi}{1}\PY{p}{]}\PY{o}{+}\PY{o}{.}\PY{l+m+mi}{1}\PY{p}{)}
\PY{n}{vxu}\PY{o}{.}\PY{n}{set\PYZus{}ylim}\PY{p}{(}\PY{n}{sol\PYZus{}v}\PY{o}{.}\PY{n}{y}\PY{p}{[}\PY{l+m+mi}{2}\PY{p}{]}\PY{o}{.}\PY{n}{min}\PY{p}{(}\PY{p}{)} \PY{o}{\PYZhy{}} \PY{l+m+mi}{1}\PY{p}{,} \PY{n}{sol\PYZus{}v}\PY{o}{.}\PY{n}{y}\PY{p}{[}\PY{l+m+mi}{2}\PY{p}{]}\PY{o}{.}\PY{n}{max}\PY{p}{(}\PY{p}{)}\PY{o}{+}\PY{l+m+mi}{1}\PY{p}{)}
\end{Verbatim}
\end{tcolorbox}

            \begin{tcolorbox}[breakable, size=fbox, boxrule=.5pt, pad at break*=1mm, opacityfill=0]
\prompt{Out}{outcolor}{15}{\boxspacing}
\begin{Verbatim}[commandchars=\\\{\}]
(0.9998848376388001, 22.213203435596427)
\end{Verbatim}
\end{tcolorbox}
        
    \begin{center}
    \adjustimage{max size={0.9\linewidth}{0.9\paperheight}}{Relatorio_files/Relatorio_31_1.png}
    \end{center}
    { \hspace*{\fill} \\}
    
    No gráfico da velocidade em função do tempo, note que este valor
colabora com o que esperávamos, pois a velocidade tende à mesma do
vento. Visto que o vento desacelerará ou acelerará o projétil por meio
da resistência do ar. Note que, como há velocidade favorável ao
movimento, a posição sempre aumentará, o que não ocorreria se não
houvesse vento ou se tivesse um vento contrário, que mudaria o sentido
da trajetória horizontal.

\hypertarget{plotando-gruxe1fico-da-posiuxe7uxe3o-vertical-em-funuxe7uxe3o-do-tempo}{%
\paragraph{Plotando gráfico da posição vertical em função do
tempo:}\label{plotando-gruxe1fico-da-posiuxe7uxe3o-vertical-em-funuxe7uxe3o-do-tempo}}

    \begin{tcolorbox}[breakable, size=fbox, boxrule=1pt, pad at break*=1mm,colback=cellbackground, colframe=cellborder]
\prompt{In}{incolor}{16}{\boxspacing}
\begin{Verbatim}[commandchars=\\\{\}]
\PY{n}{fig}\PY{p}{,} \PY{p}{(}\PY{p}{(}\PY{n}{zw}\PY{p}{)}\PY{p}{,} \PY{p}{(}\PY{n}{vzw}\PY{p}{)}\PY{p}{)} \PY{o}{=} \PY{n}{plt}\PY{o}{.}\PY{n}{subplots}\PY{p}{(}\PY{l+m+mi}{2}\PY{p}{,} \PY{l+m+mi}{1}\PY{p}{,} \PY{n}{sharex}\PY{o}{=}\PY{k+kc}{True}\PY{p}{)}
\PY{n}{zw}\PY{o}{.}\PY{n}{plot}\PY{p}{(}\PY{n}{sol\PYZus{}v}\PY{o}{.}\PY{n}{t}\PY{p}{,} \PY{n}{sol\PYZus{}v}\PY{o}{.}\PY{n}{y}\PY{p}{[}\PY{l+m+mi}{1}\PY{p}{]}\PY{p}{,} \PY{n}{c}\PY{o}{=}\PY{l+s+s2}{\PYZdq{}}\PY{l+s+s2}{b}\PY{l+s+s2}{\PYZdq{}}\PY{p}{)}
\PY{n}{zw}\PY{o}{.}\PY{n}{set\PYZus{}ylabel}\PY{p}{(}\PY{l+s+s2}{\PYZdq{}}\PY{l+s+s2}{Posição (m)}\PY{l+s+s2}{\PYZdq{}}\PY{p}{)}
\PY{n}{zw}\PY{o}{.}\PY{n}{set\PYZus{}title}\PY{p}{(}\PY{l+s+s2}{\PYZdq{}}\PY{l+s+s2}{Posição vertical em função do tempo}\PY{l+s+s2}{\PYZdq{}}\PY{p}{)}
\PY{n}{zw}\PY{o}{.}\PY{n}{set\PYZus{}ylim}\PY{p}{(}\PY{n}{sol\PYZus{}v}\PY{o}{.}\PY{n}{y}\PY{p}{[}\PY{l+m+mi}{1}\PY{p}{]}\PY{o}{.}\PY{n}{min}\PY{p}{(}\PY{p}{)} \PY{o}{\PYZhy{}} \PY{l+m+mi}{1}\PY{p}{,} \PY{n}{sol\PYZus{}v}\PY{o}{.}\PY{n}{y}\PY{p}{[}\PY{l+m+mi}{1}\PY{p}{]}\PY{o}{.}\PY{n}{max}\PY{p}{(}\PY{p}{)}\PY{o}{+}\PY{l+m+mi}{1}\PY{p}{)}

\PY{n}{vzw}\PY{o}{.}\PY{n}{plot}\PY{p}{(}\PY{n}{sol\PYZus{}v}\PY{o}{.}\PY{n}{t}\PY{p}{,} \PY{n}{sol\PYZus{}v}\PY{o}{.}\PY{n}{y}\PY{p}{[}\PY{l+m+mi}{3}\PY{p}{]}\PY{p}{,} \PY{n}{c}\PY{o}{=}\PY{l+s+s2}{\PYZdq{}}\PY{l+s+s2}{orange}\PY{l+s+s2}{\PYZdq{}}\PY{p}{)}
\PY{n}{vzw}\PY{o}{.}\PY{n}{set\PYZus{}ylabel}\PY{p}{(}\PY{l+s+sa}{r}\PY{l+s+s2}{\PYZdq{}}\PY{l+s+s2}{velocidade (\PYZdl{}}\PY{l+s+s2}{\PYZbs{}}\PY{l+s+s2}{frac}\PY{l+s+si}{\PYZob{}m\PYZcb{}}\PY{l+s+si}{\PYZob{}s\PYZcb{}}\PY{l+s+s2}{\PYZdl{})}\PY{l+s+s2}{\PYZdq{}}\PY{p}{)}
\PY{n}{vzw}\PY{o}{.}\PY{n}{set\PYZus{}title}\PY{p}{(}\PY{l+s+s2}{\PYZdq{}}\PY{l+s+s2}{Velocidade vertical em função do tempo}\PY{l+s+s2}{\PYZdq{}}\PY{p}{)}
\PY{n}{vzw}\PY{o}{.}\PY{n}{set\PYZus{}xlabel}\PY{p}{(}\PY{l+s+s2}{\PYZdq{}}\PY{l+s+s2}{Tempo (t)}\PY{l+s+s2}{\PYZdq{}}\PY{p}{)}
\PY{n}{vzw}\PY{o}{.}\PY{n}{set\PYZus{}xlim}\PY{p}{(}\PY{n}{t}\PY{p}{[}\PY{l+m+mi}{0}\PY{p}{]}\PY{o}{\PYZhy{}}\PY{o}{.}\PY{l+m+mi}{1}\PY{p}{,} \PY{n}{t}\PY{p}{[}\PY{o}{\PYZhy{}}\PY{l+m+mi}{1}\PY{p}{]}\PY{o}{+}\PY{o}{.}\PY{l+m+mi}{1}\PY{p}{)}
\PY{n}{vzw}\PY{o}{.}\PY{n}{set\PYZus{}ylim}\PY{p}{(}\PY{n}{sol\PYZus{}v}\PY{o}{.}\PY{n}{y}\PY{p}{[}\PY{l+m+mi}{3}\PY{p}{]}\PY{o}{.}\PY{n}{min}\PY{p}{(}\PY{p}{)} \PY{o}{\PYZhy{}} \PY{l+m+mi}{1}\PY{p}{,} \PY{n}{sol\PYZus{}v}\PY{o}{.}\PY{n}{y}\PY{p}{[}\PY{l+m+mi}{3}\PY{p}{]}\PY{o}{.}\PY{n}{max}\PY{p}{(}\PY{p}{)}\PY{o}{+}\PY{l+m+mi}{1}\PY{p}{)}
\end{Verbatim}
\end{tcolorbox}

            \begin{tcolorbox}[breakable, size=fbox, boxrule=.5pt, pad at break*=1mm, opacityfill=0]
\prompt{Out}{outcolor}{16}{\boxspacing}
\begin{Verbatim}[commandchars=\\\{\}]
(-2.041473987891144, 22.213203435596427)
\end{Verbatim}
\end{tcolorbox}
        
    \begin{center}
    \adjustimage{max size={0.9\linewidth}{0.9\paperheight}}{Relatorio_files/Relatorio_33_1.png}
    \end{center}
    { \hspace*{\fill} \\}
    
    Como a velocidade do vento vertical é ascentente, perceba que o projétil
desacelera mais lentamente durante a subida e no movimento descendente
mais rapidamente. Dessa forma, causará uma queda do projétil em mais
tempo tempo. Além disso, observe que a velocidade em \(z\) também
converge, pois, quanto maior a velocidade, maior será a influência da
resistência do ar. Até o ponto onde se igualarão, ou seja, a posição em
função do tempo será linear. Repare que se usarmos um vento, no qual
seja o suficiente para igualar a aceleração gravitacional com o projétil
parado, teremos um projétil levitando e sempre tendendo a zero a
velocidade vertical, pois, se tiver alguma velocidade a força da
resistencia do ar pode diminuir ou almentar tendendo a velocidade zero.

\hypertarget{alternativa-b}{%
\subsection{Alternativa (b)}\label{alternativa-b}}

Como o objetivo é pegar diversas combinações entre \(\theta\), \(u\) e
\(w\). Sabendo disso, usaremos a função \texttt{np.random.randint()}
para aleatorizar as escolhas das constantes \(u\) e \(w\) variando
\(\theta\) entre \([5; 90]\). Dessa forma, temos:

    \begin{tcolorbox}[breakable, size=fbox, boxrule=1pt, pad at break*=1mm,colback=cellbackground, colframe=cellborder]
\prompt{In}{incolor}{17}{\boxspacing}
\begin{Verbatim}[commandchars=\\\{\}]
\PY{c+c1}{\PYZsh{} Função para capturar pontos, no qual a velocidade em z é zero}
\PY{k}{def} \PY{n+nf}{topo\PYZus{}uw}\PY{p}{(}\PY{n}{t}\PY{p}{,} \PY{n}{r}\PY{p}{,} \PY{n}{g}\PY{p}{,} \PY{n}{c}\PY{p}{,} \PY{n}{m}\PY{p}{,} \PY{n}{u}\PY{p}{,} \PY{n}{w}\PY{p}{)}\PY{p}{:}
    \PY{n}{x}\PY{p}{,} \PY{n}{z}\PY{p}{,} \PY{n}{vx}\PY{p}{,} \PY{n}{vz} \PY{o}{=} \PY{n}{r}
    \PY{k}{return} \PY{n}{vz}


\PY{c+c1}{\PYZsh{} Função para capturar pontos, no qual z é zero}
\PY{k}{def} \PY{n+nf}{alcanse\PYZus{}uw}\PY{p}{(}\PY{n}{t}\PY{p}{,} \PY{n}{r}\PY{p}{,} \PY{n}{g}\PY{p}{,} \PY{n}{c}\PY{p}{,} \PY{n}{m}\PY{p}{,} \PY{n}{u}\PY{p}{,} \PY{n}{w}\PY{p}{)}\PY{p}{:}
    \PY{n}{x}\PY{p}{,} \PY{n}{z}\PY{p}{,} \PY{n}{vx}\PY{p}{,} \PY{n}{vz} \PY{o}{=} \PY{n}{r}
    \PY{k}{return} \PY{n}{z}


\PY{c+c1}{\PYZsh{} arrays para agrupar os pontos, para fazer um auto limite no grafico}
\PY{n}{chao} \PY{o}{=} \PY{n}{np}\PY{o}{.}\PY{n}{array}\PY{p}{(}\PY{p}{[}\PY{p}{]}\PY{p}{)}
\PY{n}{topo} \PY{o}{=} \PY{n}{np}\PY{o}{.}\PY{n}{array}\PY{p}{(}\PY{p}{[}\PY{p}{]}\PY{p}{)}


\PY{k}{for} \PY{n}{\PYZus{}} \PY{o+ow}{in} \PY{n}{np}\PY{o}{.}\PY{n}{arange}\PY{p}{(}\PY{l+m+mi}{5}\PY{p}{,} \PY{l+m+mi}{90}\PY{p}{,} \PY{l+m+mi}{15}\PY{p}{)}\PY{p}{:}
    \PY{n}{theta\PYZus{}i} \PY{o}{=} \PY{n}{np}\PY{o}{.}\PY{n}{radians}\PY{p}{(}\PY{n}{\PYZus{}}\PY{p}{)}
    \PY{n}{u}\PY{p}{,} \PY{n}{w} \PY{o}{=} \PY{n}{np}\PY{o}{.}\PY{n}{random}\PY{o}{.}\PY{n}{uniform}\PY{p}{(}\PY{o}{\PYZhy{}}\PY{l+m+mi}{3}\PY{p}{,} \PY{l+m+mi}{3}\PY{p}{,} \PY{l+m+mi}{2}\PY{p}{)}
    \PY{n}{vx0i} \PY{o}{=} \PY{n}{v0}\PY{o}{*}\PY{n}{np}\PY{o}{.}\PY{n}{cos}\PY{p}{(}\PY{n}{theta\PYZus{}i}\PY{p}{)}
    \PY{n}{vz0i} \PY{o}{=} \PY{n}{v0}\PY{o}{*}\PY{n}{np}\PY{o}{.}\PY{n}{sin}\PY{p}{(}\PY{n}{theta\PYZus{}i}\PY{p}{)}
    \PY{n}{sol\PYZus{}vi} \PY{o}{=} \PY{n}{solve\PYZus{}ivp}\PY{p}{(}\PY{n}{fun}\PY{o}{=}\PY{n}{r\PYZus{}v}\PY{p}{,} \PY{n}{t\PYZus{}span}\PY{o}{=}\PY{p}{[}\PY{n}{t}\PY{p}{[}\PY{l+m+mi}{0}\PY{p}{]}\PY{p}{,} \PY{n}{t}\PY{p}{[}\PY{o}{\PYZhy{}}\PY{l+m+mi}{1}\PY{p}{]}\PY{p}{]}\PY{p}{,} \PY{n}{y0}\PY{o}{=}\PY{p}{[}\PY{n}{x0}\PY{p}{,} \PY{n}{z0}\PY{p}{,} \PY{n}{vx0i}\PY{p}{,} \PY{n}{vz0i}\PY{p}{]}\PY{p}{,} \PY{n}{args}\PY{o}{=}\PY{p}{(}\PY{n}{g}\PY{p}{,} \PY{n}{c}\PY{p}{,} \PY{n}{m}\PY{p}{,} \PY{n}{u}\PY{p}{,} \PY{n}{w}\PY{p}{)}\PY{p}{,} \PY{n}{t\PYZus{}eval}\PY{o}{=}\PY{n}{t}\PY{p}{,} \PY{n}{method}\PY{o}{=}\PY{l+s+s2}{\PYZdq{}}\PY{l+s+s2}{LSODA}\PY{l+s+s2}{\PYZdq{}}\PY{p}{,} \PY{n}{events}\PY{o}{=}\PY{p}{(}\PY{n}{alcanse\PYZus{}uw}\PY{p}{,} \PY{n}{topo\PYZus{}uw}\PY{p}{)}\PY{p}{)}
    \PY{c+c1}{\PYZsh{} Recolhendo os pontos}
    \PY{n}{chao} \PY{o}{=} \PY{n}{np}\PY{o}{.}\PY{n}{append}\PY{p}{(}\PY{n}{chao}\PY{p}{,} \PY{n}{sol\PYZus{}vi}\PY{o}{.}\PY{n}{y\PYZus{}events}\PY{p}{[}\PY{l+m+mi}{0}\PY{p}{]}\PY{p}{[}\PY{p}{:}\PY{p}{,} \PY{l+m+mi}{0}\PY{p}{]}\PY{p}{)}
    \PY{n}{topo} \PY{o}{=} \PY{n}{np}\PY{o}{.}\PY{n}{append}\PY{p}{(}\PY{n}{topo}\PY{p}{,} \PY{n}{sol\PYZus{}vi}\PY{o}{.}\PY{n}{y\PYZus{}events}\PY{p}{[}\PY{l+m+mi}{1}\PY{p}{]}\PY{p}{[}\PY{p}{:}\PY{p}{,} \PY{l+m+mi}{1}\PY{p}{]}\PY{p}{)}
    \PY{n}{plt}\PY{o}{.}\PY{n}{plot}\PY{p}{(}\PY{n}{sol\PYZus{}vi}\PY{o}{.}\PY{n}{y}\PY{p}{[}\PY{l+m+mi}{0}\PY{p}{]}\PY{p}{,} \PY{n}{sol\PYZus{}vi}\PY{o}{.}\PY{n}{y}\PY{p}{[}\PY{l+m+mi}{1}\PY{p}{]}\PY{p}{,} \PY{n}{label}\PY{o}{=}\PY{l+s+sa}{r}\PY{l+s+s2}{\PYZdq{}}\PY{l+s+s2}{\PYZdl{}u=\PYZdl{}}\PY{l+s+s2}{\PYZdq{}}\PY{l+s+sa}{f}\PY{l+s+s2}{\PYZdq{}}\PY{l+s+si}{\PYZob{}}\PY{n}{u}\PY{l+s+si}{:}\PY{l+s+s2}{.2f}\PY{l+s+si}{\PYZcb{}}\PY{l+s+s2}{,}\PY{l+s+s2}{\PYZdq{}}\PY{o}{+}\PY{l+s+sa}{r}\PY{l+s+s2}{\PYZdq{}}\PY{l+s+s2}{ \PYZdl{}w=\PYZdl{}}\PY{l+s+s2}{\PYZdq{}}\PY{l+s+sa}{f}\PY{l+s+s2}{\PYZdq{}}\PY{l+s+si}{\PYZob{}}\PY{n}{w}\PY{l+s+si}{:}\PY{l+s+s2}{.2f}\PY{l+s+si}{\PYZcb{}}\PY{l+s+s2}{\PYZdq{}}\PY{p}{,} \PY{n}{c}\PY{o}{=}\PY{n}{cmap}\PY{o}{.}\PY{n}{to\PYZus{}rgba}\PY{p}{(}\PY{n}{\PYZus{}} \PY{o}{+} \PY{l+m+mi}{1}\PY{p}{)}\PY{p}{)}

\PY{n}{plt}\PY{o}{.}\PY{n}{colorbar}\PY{p}{(}\PY{n}{cmap}\PY{p}{,} \PY{n}{ticks}\PY{o}{=}\PY{n}{np}\PY{o}{.}\PY{n}{arange}\PY{p}{(}\PY{l+m+mi}{5}\PY{p}{,} \PY{l+m+mi}{95}\PY{p}{,} \PY{l+m+mi}{10}\PY{p}{)}\PY{p}{,} \PY{n}{label}\PY{o}{=}\PY{l+s+sa}{r}\PY{l+s+s2}{\PYZdq{}}\PY{l+s+s2}{ângulos (\PYZdl{}}\PY{l+s+s2}{\PYZbs{}}\PY{l+s+s2}{theta\PYZdl{}°)}\PY{l+s+s2}{\PYZdq{}}\PY{p}{)}
\PY{c+c1}{\PYZsh{} usando os pontos para ajustar plotagem automatico}
\PY{n}{plt}\PY{o}{.}\PY{n}{xlim}\PY{p}{(}\PY{n}{chao}\PY{o}{.}\PY{n}{min}\PY{p}{(}\PY{p}{)}\PY{o}{\PYZhy{}}\PY{o}{.}\PY{l+m+mi}{2}\PY{p}{,} \PY{n}{chao}\PY{o}{.}\PY{n}{max}\PY{p}{(}\PY{p}{)}\PY{o}{+}\PY{o}{.}\PY{l+m+mi}{2}\PY{p}{)}
\PY{n}{plt}\PY{o}{.}\PY{n}{ylim}\PY{p}{(}\PY{l+m+mi}{0}\PY{p}{,} \PY{n}{topo}\PY{o}{.}\PY{n}{max}\PY{p}{(}\PY{p}{)}\PY{o}{+}\PY{o}{.}\PY{l+m+mi}{1}\PY{p}{)}
\PY{n}{plt}\PY{o}{.}\PY{n}{title}\PY{p}{(}\PY{l+s+s2}{\PYZdq{}}\PY{l+s+s2}{Posição vertical em função da horizontal}\PY{l+s+s2}{\PYZdq{}}\PY{p}{)}
\PY{n}{plt}\PY{o}{.}\PY{n}{xlabel}\PY{p}{(}\PY{l+s+s2}{\PYZdq{}}\PY{l+s+s2}{Horizontal (m)}\PY{l+s+s2}{\PYZdq{}}\PY{p}{)}
\PY{n}{plt}\PY{o}{.}\PY{n}{ylabel}\PY{p}{(}\PY{l+s+s2}{\PYZdq{}}\PY{l+s+s2}{Vertical (m)}\PY{l+s+s2}{\PYZdq{}}\PY{p}{)}
\PY{n}{plt}\PY{o}{.}\PY{n}{legend}\PY{p}{(}\PY{p}{)}
\end{Verbatim}
\end{tcolorbox}

            \begin{tcolorbox}[breakable, size=fbox, boxrule=.5pt, pad at break*=1mm, opacityfill=0]
\prompt{Out}{outcolor}{17}{\boxspacing}
\begin{Verbatim}[commandchars=\\\{\}]
<matplotlib.legend.Legend at 0x7fb5ea173be0>
\end{Verbatim}
\end{tcolorbox}
        
    \begin{center}
    \adjustimage{max size={0.9\linewidth}{0.9\paperheight}}{Relatorio_files/Relatorio_35_1.png}
    \end{center}
    { \hspace*{\fill} \\}
    
    Note que, para ângulos menores o alcance e altura são mais consistentes.
Porque, quanto menos tempo o projétil fica no ar, menos tempo ele será
influenciado pela resistência do ar. Além disso, como as entradas de
velocidade de vento são aleatórias dentro um intervalo de \([-3; 3]\) há
momentos que o vento corrobora com a velocidade vertical, de modo que
mantém o projétil por mais tempo no ar. Dessa forma, fazendo com que o
vento horizontal leve o projétil para pontos de lançamento, ou antes.
Contudo, isso só ocorre para ângulos maiores, pois, há menos velocidade
horizontal e o projétil permanece mais tempo no ar.

Com o link do GitHub pode-se testar esta geração randomizada. Note que,
o gráfico se ajusta automaticamente para a melhor representação.

\hypertarget{questuxe3o-3}{%
\section{Questão 3}\label{questuxe3o-3}}

Semelhante às questões anteriores, porém agora com a resistência do ar
variando. Ou seja, teremos um comportamento diferente, pois quanto mais
alto a partícula estiver, menor será a resistência do ar. E com isto,
freando menos.

O fator com que a resistência do ar diminui é descrita por:
\[e^{-\frac{z}{h}}\]

As condições iniciais são as mesmas.

\hypertarget{alternativa-a}{%
\subsection{Alternativa (a)}\label{alternativa-a}}

Como a resistência do ar varia em função da altura precisamos aplicar em
ambas as equações do movimento.

\[ \frac{d^2x}{dt^2} = - \frac{c}{m} \frac{dx}{dt}\sqrt{\left(\frac{dx}{dt} \right)^2 + \left(\frac{dz}{dt} \right)^2} e^{-z/h}
\]
\[\frac{d^2z}{dt^2} = - g - \frac{c}{m} \frac{dz}{dt}\sqrt{\left(\frac{dx}{dt} \right)^2 + \left(\frac{dz}{dt} \right)^2} e^{-z/h}
\]

\hypertarget{definindo-funuxe7uxe3o-par-resolver-as-edos}{%
\paragraph{Definindo função par resolver as
EDOs:}\label{definindo-funuxe7uxe3o-par-resolver-as-edos}}

    \begin{tcolorbox}[breakable, size=fbox, boxrule=1pt, pad at break*=1mm,colback=cellbackground, colframe=cellborder]
\prompt{In}{incolor}{18}{\boxspacing}
\begin{Verbatim}[commandchars=\\\{\}]
\PY{k}{def} \PY{n+nf}{r\PYZus{}h}\PY{p}{(}\PY{n}{t}\PY{p}{,} \PY{n}{r}\PY{p}{,} \PY{n}{g}\PY{p}{,} \PY{n}{c}\PY{p}{,} \PY{n}{m}\PY{p}{,} \PY{n}{h}\PY{p}{)}\PY{p}{:}
    \PY{n}{x}\PY{p}{,} \PY{n}{z}\PY{p}{,} \PY{n}{vx}\PY{p}{,} \PY{n}{vz} \PY{o}{=} \PY{n}{r}
    \PY{n}{ddx} \PY{o}{=} \PY{o}{\PYZhy{}} \PY{n}{c}\PY{o}{/}\PY{n}{m} \PY{o}{*} \PY{n}{vx} \PY{o}{*} \PY{n}{np}\PY{o}{.}\PY{n}{exp}\PY{p}{(}\PY{o}{\PYZhy{}}\PY{n}{z}\PY{o}{/}\PY{n}{h}\PY{p}{)} \PY{o}{*} \PY{n}{np}\PY{o}{.}\PY{n}{hypot}\PY{p}{(}\PY{n}{x}\PY{p}{,} \PY{n}{z}\PY{p}{)}
    \PY{n}{ddz} \PY{o}{=} \PY{o}{\PYZhy{}}\PY{n}{g} \PY{o}{\PYZhy{}} \PY{n}{c}\PY{o}{/}\PY{n}{m} \PY{o}{*} \PY{n}{vz} \PY{o}{*} \PY{n}{np}\PY{o}{.}\PY{n}{exp}\PY{p}{(}\PY{o}{\PYZhy{}}\PY{n}{z}\PY{o}{/}\PY{n}{h}\PY{p}{)} \PY{o}{*} \PY{n}{np}\PY{o}{.}\PY{n}{hypot}\PY{p}{(}\PY{n}{x}\PY{p}{,} \PY{n}{z}\PY{p}{)}
    \PY{k}{return} \PY{p}{[}\PY{n}{vx}\PY{p}{,} \PY{n}{vz}\PY{p}{,} \PY{n}{ddx}\PY{p}{,} \PY{n}{ddz}\PY{p}{]}
\end{Verbatim}
\end{tcolorbox}

    \hypertarget{definindo-condiuxe7uxf5es-iniciais-constantes-intervalo-da-soluuxe7uxe3o-numerica}{%
\paragraph{Definindo condições iniciais, constantes, intervalo da
solução
numerica:}\label{definindo-condiuxe7uxf5es-iniciais-constantes-intervalo-da-soluuxe7uxe3o-numerica}}

    \begin{tcolorbox}[breakable, size=fbox, boxrule=1pt, pad at break*=1mm,colback=cellbackground, colframe=cellborder]
\prompt{In}{incolor}{19}{\boxspacing}
\begin{Verbatim}[commandchars=\\\{\}]
\PY{c+c1}{\PYZsh{} condições iniciais}
\PY{n}{x0} \PY{o}{=} \PY{l+m+mi}{0}
\PY{n}{z0} \PY{o}{=} \PY{l+m+mi}{0}
\PY{n}{v0} \PY{o}{=} \PY{l+m+mi}{30}
\PY{n}{theta} \PY{o}{=} \PY{n}{np}\PY{o}{.}\PY{n}{radians}\PY{p}{(}\PY{l+m+mi}{45}\PY{p}{)}
\PY{n}{vx0} \PY{o}{=} \PY{n}{v0}\PY{o}{*}\PY{n}{np}\PY{o}{.}\PY{n}{cos}\PY{p}{(}\PY{n}{theta}\PY{p}{)}
\PY{n}{vz0} \PY{o}{=} \PY{n}{v0}\PY{o}{*}\PY{n}{np}\PY{o}{.}\PY{n}{sin}\PY{p}{(}\PY{n}{theta}\PY{p}{)}
\PY{c+c1}{\PYZsh{} constantes}
\PY{n}{g} \PY{o}{=} \PY{l+m+mf}{9.8}
\PY{n}{c} \PY{o}{=} \PY{l+m+mf}{1.3}
\PY{n}{m} \PY{o}{=} \PY{l+m+mi}{5}
\PY{n}{h} \PY{o}{=} \PY{l+m+mi}{4}
\PY{c+c1}{\PYZsh{} Intervalo da solução}
\PY{n}{t} \PY{o}{=} \PY{n}{np}\PY{o}{.}\PY{n}{linspace}\PY{p}{(}\PY{l+m+mi}{0}\PY{p}{,} \PY{l+m+mi}{7}\PY{p}{,} \PY{l+m+mi}{200}\PY{p}{,} \PY{k+kc}{True}\PY{p}{)}
\end{Verbatim}
\end{tcolorbox}

    \hypertarget{resolvendo-a-edo-numericamente-com-o-muxe9todo-lsoda}{%
\paragraph{Resolvendo a EDO numericamente com o método
LSODA:}\label{resolvendo-a-edo-numericamente-com-o-muxe9todo-lsoda}}

    \begin{tcolorbox}[breakable, size=fbox, boxrule=1pt, pad at break*=1mm,colback=cellbackground, colframe=cellborder]
\prompt{In}{incolor}{20}{\boxspacing}
\begin{Verbatim}[commandchars=\\\{\}]
\PY{n}{sol\PYZus{}e} \PY{o}{=} \PY{n}{solve\PYZus{}ivp}\PY{p}{(}\PY{n}{fun}\PY{o}{=}\PY{n}{r\PYZus{}h}\PY{p}{,} \PY{n}{t\PYZus{}span}\PY{o}{=}\PY{p}{[}\PY{n}{t}\PY{p}{[}\PY{l+m+mi}{0}\PY{p}{]}\PY{p}{,} \PY{n}{t}\PY{p}{[}\PY{o}{\PYZhy{}}\PY{l+m+mi}{1}\PY{p}{]}\PY{p}{]}\PY{p}{,} \PY{n}{y0}\PY{o}{=}\PY{p}{[}\PY{n}{x0}\PY{p}{,} \PY{n}{x0}\PY{p}{,} \PY{n}{vx0}\PY{p}{,} \PY{n}{vz0}\PY{p}{]}\PY{p}{,} \PY{n}{method}\PY{o}{=}\PY{l+s+s2}{\PYZdq{}}\PY{l+s+s2}{LSODA}\PY{l+s+s2}{\PYZdq{}}\PY{p}{,} \PY{n}{t\PYZus{}eval}\PY{o}{=}\PY{n}{t}\PY{p}{,} \PY{n}{args}\PY{o}{=}\PY{p}{(}\PY{n}{g}\PY{p}{,} \PY{n}{c}\PY{p}{,} \PY{n}{m}\PY{p}{,} \PY{n}{h}\PY{p}{)}\PY{p}{)}
\end{Verbatim}
\end{tcolorbox}

    \hypertarget{plotando-gruxe1fico-da-posiuxe7uxe3o-horizontal-em-funuxe7uxe3o-do-tempo}{%
\paragraph{Plotando gráfico da posição horizontal em função do
tempo:}\label{plotando-gruxe1fico-da-posiuxe7uxe3o-horizontal-em-funuxe7uxe3o-do-tempo}}

    \begin{tcolorbox}[breakable, size=fbox, boxrule=1pt, pad at break*=1mm,colback=cellbackground, colframe=cellborder]
\prompt{In}{incolor}{21}{\boxspacing}
\begin{Verbatim}[commandchars=\\\{\}]
\PY{n}{fig}\PY{p}{,} \PY{p}{(}\PY{p}{(}\PY{n}{x}\PY{p}{)}\PY{p}{,} \PY{p}{(}\PY{n}{vx}\PY{p}{)}\PY{p}{)} \PY{o}{=} \PY{n}{plt}\PY{o}{.}\PY{n}{subplots}\PY{p}{(}\PY{l+m+mi}{2}\PY{p}{,} \PY{l+m+mi}{1}\PY{p}{,} \PY{n}{sharex}\PY{o}{=}\PY{k+kc}{True}\PY{p}{)}
\PY{n}{x}\PY{o}{.}\PY{n}{plot}\PY{p}{(}\PY{n}{sol\PYZus{}e}\PY{o}{.}\PY{n}{t}\PY{p}{,} \PY{n}{sol\PYZus{}e}\PY{o}{.}\PY{n}{y}\PY{p}{[}\PY{l+m+mi}{0}\PY{p}{]}\PY{p}{,} \PY{n}{label}\PY{o}{=}\PY{l+s+s2}{\PYZdq{}}\PY{l+s+s2}{Posição em fução do tempo}\PY{l+s+s2}{\PYZdq{}}\PY{p}{,} \PY{n}{c}\PY{o}{=}\PY{l+s+s2}{\PYZdq{}}\PY{l+s+s2}{b}\PY{l+s+s2}{\PYZdq{}}\PY{p}{)}
\PY{n}{x}\PY{o}{.}\PY{n}{set\PYZus{}ylabel}\PY{p}{(}\PY{l+s+s2}{\PYZdq{}}\PY{l+s+s2}{Posição (m)}\PY{l+s+s2}{\PYZdq{}}\PY{p}{)}
\PY{n}{x}\PY{o}{.}\PY{n}{set\PYZus{}title}\PY{p}{(}\PY{l+s+s2}{\PYZdq{}}\PY{l+s+s2}{Posição em função do tempo}\PY{l+s+s2}{\PYZdq{}}\PY{p}{)}
\PY{n}{x}\PY{o}{.}\PY{n}{set\PYZus{}ylim}\PY{p}{(}\PY{n}{sol\PYZus{}e}\PY{o}{.}\PY{n}{y}\PY{p}{[}\PY{l+m+mi}{0}\PY{p}{]}\PY{o}{.}\PY{n}{min}\PY{p}{(}\PY{p}{)}\PY{o}{\PYZhy{}}\PY{l+m+mi}{2}\PY{p}{,} \PY{n}{sol\PYZus{}e}\PY{o}{.}\PY{n}{y}\PY{p}{[}\PY{l+m+mi}{0}\PY{p}{]}\PY{o}{.}\PY{n}{max}\PY{p}{(}\PY{p}{)}\PY{o}{+}\PY{l+m+mi}{2}\PY{p}{)}

\PY{n}{vx}\PY{o}{.}\PY{n}{plot}\PY{p}{(}\PY{n}{sol\PYZus{}e}\PY{o}{.}\PY{n}{t}\PY{p}{,} \PY{n}{sol\PYZus{}e}\PY{o}{.}\PY{n}{y}\PY{p}{[}\PY{l+m+mi}{2}\PY{p}{]}\PY{p}{,} \PY{n}{label}\PY{o}{=}\PY{l+s+s2}{\PYZdq{}}\PY{l+s+s2}{Velocidade em função do tempo}\PY{l+s+s2}{\PYZdq{}}\PY{p}{,} \PY{n}{c}\PY{o}{=}\PY{l+s+s2}{\PYZdq{}}\PY{l+s+s2}{orange}\PY{l+s+s2}{\PYZdq{}}\PY{p}{)}
\PY{n}{vx}\PY{o}{.}\PY{n}{set\PYZus{}xlabel}\PY{p}{(}\PY{l+s+s2}{\PYZdq{}}\PY{l+s+s2}{Tempo (s)}\PY{l+s+s2}{\PYZdq{}}\PY{p}{)}
\PY{n}{vx}\PY{o}{.}\PY{n}{set\PYZus{}ylabel}\PY{p}{(}\PY{l+s+sa}{r}\PY{l+s+s2}{\PYZdq{}}\PY{l+s+s2}{Velocidade (\PYZdl{}}\PY{l+s+s2}{\PYZbs{}}\PY{l+s+s2}{frac}\PY{l+s+si}{\PYZob{}m\PYZcb{}}\PY{l+s+si}{\PYZob{}s\PYZcb{}}\PY{l+s+s2}{\PYZdl{})}\PY{l+s+s2}{\PYZdq{}}\PY{p}{)}
\PY{n}{vx}\PY{o}{.}\PY{n}{set\PYZus{}title}\PY{p}{(}\PY{l+s+s2}{\PYZdq{}}\PY{l+s+s2}{Velocidade em função do tempo}\PY{l+s+s2}{\PYZdq{}}\PY{p}{)}
\PY{n}{vx}\PY{o}{.}\PY{n}{set\PYZus{}xlim}\PY{p}{(}\PY{n}{t}\PY{p}{[}\PY{l+m+mi}{0}\PY{p}{]}\PY{o}{\PYZhy{}}\PY{o}{.}\PY{l+m+mi}{1}\PY{p}{,} \PY{n}{t}\PY{p}{[}\PY{o}{\PYZhy{}}\PY{l+m+mi}{1}\PY{p}{]}\PY{o}{+}\PY{o}{.}\PY{l+m+mi}{1}\PY{p}{)}
\PY{n}{vx}\PY{o}{.}\PY{n}{set\PYZus{}ylim}\PY{p}{(}\PY{n}{sol\PYZus{}e}\PY{o}{.}\PY{n}{y}\PY{p}{[}\PY{l+m+mi}{2}\PY{p}{]}\PY{o}{.}\PY{n}{min}\PY{p}{(}\PY{p}{)}\PY{o}{\PYZhy{}}\PY{l+m+mi}{1}\PY{p}{,} \PY{n}{sol\PYZus{}e}\PY{o}{.}\PY{n}{y}\PY{p}{[}\PY{l+m+mi}{2}\PY{p}{]}\PY{o}{.}\PY{n}{max}\PY{p}{(}\PY{p}{)}\PY{o}{+}\PY{l+m+mi}{1}\PY{p}{)}
\end{Verbatim}
\end{tcolorbox}

            \begin{tcolorbox}[breakable, size=fbox, boxrule=.5pt, pad at break*=1mm, opacityfill=0]
\prompt{Out}{outcolor}{21}{\boxspacing}
\begin{Verbatim}[commandchars=\\\{\}]
(-1.000000526476611, 22.213203435596427)
\end{Verbatim}
\end{tcolorbox}
        
    \begin{center}
    \adjustimage{max size={0.9\linewidth}{0.9\paperheight}}{Relatorio_files/Relatorio_43_1.png}
    \end{center}
    { \hspace*{\fill} \\}
    
    Observe que o comportamento é quase linear, isto ocorre, pois, quanto
mais alto, menor será a resistência do ar. Com a diminuição da
velocidade, a influência da resistência também diminui. Dessa forma,
atenuando a curva por dois motivos. Além disso, o projétil, ao realizar
o movimento descendente, tem a resistência do ar aumentada
exponencialmente. Aponto em casos de parar o projétil no eixo \(x\)
antes de tocar o chão, pois com o aumento exponencial da resistência do
ar a velocidade tende a zero cada vez mais rapidamente.

\hypertarget{plotando-gruxe1fico-da-posiuxe7uxe3o-vertical-em-funuxe7uxe3o-do-tempo}{%
\paragraph{Plotando gráfico da posição vertical em função do
tempo:}\label{plotando-gruxe1fico-da-posiuxe7uxe3o-vertical-em-funuxe7uxe3o-do-tempo}}

    \begin{tcolorbox}[breakable, size=fbox, boxrule=1pt, pad at break*=1mm,colback=cellbackground, colframe=cellborder]
\prompt{In}{incolor}{22}{\boxspacing}
\begin{Verbatim}[commandchars=\\\{\}]
\PY{n}{fig}\PY{p}{,} \PY{p}{(}\PY{p}{(}\PY{n}{z}\PY{p}{)}\PY{p}{,} \PY{p}{(}\PY{n}{vz}\PY{p}{)}\PY{p}{)} \PY{o}{=} \PY{n}{plt}\PY{o}{.}\PY{n}{subplots}\PY{p}{(}\PY{l+m+mi}{2}\PY{p}{,} \PY{l+m+mi}{1}\PY{p}{,} \PY{n}{sharex}\PY{o}{=}\PY{k+kc}{True}\PY{p}{)}
\PY{n}{z}\PY{o}{.}\PY{n}{plot}\PY{p}{(}\PY{n}{sol\PYZus{}e}\PY{o}{.}\PY{n}{t}\PY{p}{,} \PY{n}{sol\PYZus{}e}\PY{o}{.}\PY{n}{y}\PY{p}{[}\PY{l+m+mi}{1}\PY{p}{]}\PY{p}{,} \PY{n}{c}\PY{o}{=}\PY{l+s+s2}{\PYZdq{}}\PY{l+s+s2}{b}\PY{l+s+s2}{\PYZdq{}}\PY{p}{)}
\PY{n}{z}\PY{o}{.}\PY{n}{set\PYZus{}ylabel}\PY{p}{(}\PY{l+s+s2}{\PYZdq{}}\PY{l+s+s2}{Posição (m)}\PY{l+s+s2}{\PYZdq{}}\PY{p}{)}
\PY{n}{z}\PY{o}{.}\PY{n}{set\PYZus{}title}\PY{p}{(}\PY{l+s+s2}{\PYZdq{}}\PY{l+s+s2}{Posição vertical em função do tempo}\PY{l+s+s2}{\PYZdq{}}\PY{p}{)}
\PY{n}{z}\PY{o}{.}\PY{n}{set\PYZus{}ylim}\PY{p}{(}\PY{n}{sol\PYZus{}e}\PY{o}{.}\PY{n}{y}\PY{p}{[}\PY{l+m+mi}{1}\PY{p}{]}\PY{o}{.}\PY{n}{min}\PY{p}{(}\PY{p}{)}\PY{o}{\PYZhy{}}\PY{l+m+mi}{1}\PY{p}{,} \PY{n}{sol\PYZus{}e}\PY{o}{.}\PY{n}{y}\PY{p}{[}\PY{l+m+mi}{1}\PY{p}{]}\PY{o}{.}\PY{n}{max}\PY{p}{(}\PY{p}{)}\PY{o}{+}\PY{l+m+mi}{1}\PY{p}{)}

\PY{n}{vz}\PY{o}{.}\PY{n}{plot}\PY{p}{(}\PY{n}{sol\PYZus{}e}\PY{o}{.}\PY{n}{t}\PY{p}{,} \PY{n}{sol\PYZus{}e}\PY{o}{.}\PY{n}{y}\PY{p}{[}\PY{l+m+mi}{3}\PY{p}{]}\PY{p}{,} \PY{n}{c}\PY{o}{=}\PY{l+s+s2}{\PYZdq{}}\PY{l+s+s2}{orange}\PY{l+s+s2}{\PYZdq{}}\PY{p}{)}
\PY{n}{vz}\PY{o}{.}\PY{n}{set\PYZus{}ylabel}\PY{p}{(}\PY{l+s+sa}{r}\PY{l+s+s2}{\PYZdq{}}\PY{l+s+s2}{Velocidade (\PYZdl{}}\PY{l+s+s2}{\PYZbs{}}\PY{l+s+s2}{frac}\PY{l+s+si}{\PYZob{}m\PYZcb{}}\PY{l+s+si}{\PYZob{}s\PYZcb{}}\PY{l+s+s2}{\PYZdl{})}\PY{l+s+s2}{\PYZdq{}}\PY{p}{)}
\PY{n}{vz}\PY{o}{.}\PY{n}{set\PYZus{}title}\PY{p}{(}\PY{l+s+s2}{\PYZdq{}}\PY{l+s+s2}{Velocidade vertical em função do tempo}\PY{l+s+s2}{\PYZdq{}}\PY{p}{)}
\PY{n}{vz}\PY{o}{.}\PY{n}{set\PYZus{}xlabel}\PY{p}{(}\PY{l+s+s2}{\PYZdq{}}\PY{l+s+s2}{Tempo (t)}\PY{l+s+s2}{\PYZdq{}}\PY{p}{)}
\PY{n}{vz}\PY{o}{.}\PY{n}{set\PYZus{}xlim}\PY{p}{(}\PY{n}{t}\PY{p}{[}\PY{l+m+mi}{0}\PY{p}{]}\PY{o}{\PYZhy{}}\PY{o}{.}\PY{l+m+mi}{1}\PY{p}{,} \PY{n}{t}\PY{p}{[}\PY{o}{\PYZhy{}}\PY{l+m+mi}{1}\PY{p}{]}\PY{o}{+}\PY{o}{.}\PY{l+m+mi}{1}\PY{p}{)}
\PY{n}{vz}\PY{o}{.}\PY{n}{set\PYZus{}ylim}\PY{p}{(}\PY{n}{sol\PYZus{}e}\PY{o}{.}\PY{n}{y}\PY{p}{[}\PY{l+m+mi}{3}\PY{p}{]}\PY{o}{.}\PY{n}{min}\PY{p}{(}\PY{p}{)}\PY{o}{\PYZhy{}}\PY{l+m+mi}{1}\PY{p}{,} \PY{n}{sol\PYZus{}e}\PY{o}{.}\PY{n}{y}\PY{p}{[}\PY{l+m+mi}{3}\PY{p}{]}\PY{o}{.}\PY{n}{max}\PY{p}{(}\PY{p}{)}\PY{o}{+}\PY{l+m+mi}{1}\PY{p}{)}
\end{Verbatim}
\end{tcolorbox}

            \begin{tcolorbox}[breakable, size=fbox, boxrule=.5pt, pad at break*=1mm, opacityfill=0]
\prompt{Out}{outcolor}{22}{\boxspacing}
\begin{Verbatim}[commandchars=\\\{\}]
(-9.39469841116954, 22.213203435596423)
\end{Verbatim}
\end{tcolorbox}
        
    \begin{center}
    \adjustimage{max size={0.9\linewidth}{0.9\paperheight}}{Relatorio_files/Relatorio_45_1.png}
    \end{center}
    { \hspace*{\fill} \\}
    
    Note que a subida do projetil é mais rápida que a descida. Porque ao
subir com a velocidade inicial a resistência do ar diminui e para descer
iniciando verticalmente parada a resistência aumenta tanto por estar
caindo quanto por estar aumentando a velocidade. Dessa forma, atenuando
a descida. Isso fica evidente no gráfico da velocidade, pois logo após
começar a descida a velocidade é atenuada e tende a zero. Mas só ficará
próximo de zero após \(z=0\) (o que não tem significado físico), pois a
resistência do ar explodirá exponencialmente.

    \hypertarget{alternativa-b}{%
\subsubsection{Alternativa (b)}\label{alternativa-b}}

    \begin{tcolorbox}[breakable, size=fbox, boxrule=1pt, pad at break*=1mm,colback=cellbackground, colframe=cellborder]
\prompt{In}{incolor}{23}{\boxspacing}
\begin{Verbatim}[commandchars=\\\{\}]
\PY{c+c1}{\PYZsh{} Lista de estilos de linha para diferencialas por angulo}
\PY{n}{selectstyle} \PY{o}{=} \PY{p}{[}\PY{l+s+s2}{\PYZdq{}}\PY{l+s+s2}{solid}\PY{l+s+s2}{\PYZdq{}}\PY{p}{,} \PY{l+s+s2}{\PYZdq{}}\PY{l+s+s2}{dashed}\PY{l+s+s2}{\PYZdq{}}\PY{p}{,} \PY{l+s+s2}{\PYZdq{}}\PY{l+s+s2}{dotted}\PY{l+s+s2}{\PYZdq{}}\PY{p}{]}

\PY{c+c1}{\PYZsh{} loop variando theta}
\PY{k}{for} \PY{n}{\PYZus{}} \PY{o+ow}{in} \PY{n}{np}\PY{o}{.}\PY{n}{linspace}\PY{p}{(}\PY{l+m+mi}{20}\PY{p}{,} \PY{l+m+mi}{70}\PY{p}{,} \PY{l+m+mi}{3}\PY{p}{,} \PY{k+kc}{True}\PY{p}{)}\PY{p}{:}
    \PY{c+c1}{\PYZsh{} loop variando h para cada theta}
    \PY{k}{for} \PY{n}{\PYZus{}\PYZus{}} \PY{o+ow}{in} \PY{n}{np}\PY{o}{.}\PY{n}{arange}\PY{p}{(}\PY{l+m+mi}{1}\PY{p}{,} \PY{l+m+mi}{4}\PY{p}{)}\PY{p}{:}
        \PY{n}{h} \PY{o}{=} \PY{n}{\PYZus{}\PYZus{}}
        \PY{n}{theta\PYZus{}i} \PY{o}{=} \PY{n}{np}\PY{o}{.}\PY{n}{radians}\PY{p}{(}\PY{n}{\PYZus{}}\PY{p}{)}
        \PY{n}{vx0i} \PY{o}{=} \PY{n}{v0}\PY{o}{*}\PY{n}{np}\PY{o}{.}\PY{n}{cos}\PY{p}{(}\PY{n}{theta\PYZus{}i}\PY{p}{)}
        \PY{n}{vz0i} \PY{o}{=} \PY{n}{v0}\PY{o}{*}\PY{n}{np}\PY{o}{.}\PY{n}{sin}\PY{p}{(}\PY{n}{theta\PYZus{}i}\PY{p}{)}
        \PY{n}{sol\PYZus{}ei} \PY{o}{=} \PY{n}{solve\PYZus{}ivp}\PY{p}{(}\PY{n}{fun}\PY{o}{=}\PY{n}{r\PYZus{}h}\PY{p}{,} \PY{n}{t\PYZus{}span}\PY{o}{=}\PY{p}{[}\PY{n}{t}\PY{p}{[}\PY{l+m+mi}{0}\PY{p}{]}\PY{p}{,} \PY{n}{t}\PY{p}{[}\PY{o}{\PYZhy{}}\PY{l+m+mi}{1}\PY{p}{]}\PY{p}{]}\PY{p}{,} \PY{n}{y0}\PY{o}{=}\PY{p}{[}\PY{n}{x0}\PY{p}{,} \PY{n}{z0}\PY{p}{,} \PY{n}{vx0i}\PY{p}{,} \PY{n}{vz0i}\PY{p}{]}\PY{p}{,} \PY{n}{method}\PY{o}{=}\PY{l+s+s2}{\PYZdq{}}\PY{l+s+s2}{LSODA}\PY{l+s+s2}{\PYZdq{}}\PY{p}{,}\PY{n}{t\PYZus{}eval}\PY{o}{=}\PY{n}{t}\PY{p}{,} \PY{n}{args}\PY{o}{=}\PY{p}{(}\PY{n}{g}\PY{p}{,} \PY{n}{c}\PY{p}{,} \PY{n}{m}\PY{p}{,} \PY{n}{h}\PY{p}{)}\PY{p}{)}
        \PY{n}{plt}\PY{o}{.}\PY{n}{plot}\PY{p}{(}\PY{n}{sol\PYZus{}ei}\PY{o}{.}\PY{n}{y}\PY{p}{[}\PY{l+m+mi}{0}\PY{p}{]}\PY{p}{,} \PY{n}{sol\PYZus{}ei}\PY{o}{.}\PY{n}{y}\PY{p}{[}\PY{l+m+mi}{1}\PY{p}{]}\PY{p}{,} \PY{n}{label}\PY{o}{=}\PY{l+s+sa}{r}\PY{l+s+s2}{\PYZdq{}}\PY{l+s+s2}{\PYZdl{}}\PY{l+s+s2}{\PYZbs{}}\PY{l+s+s2}{theta \PYZus{}i=\PYZdl{}}\PY{l+s+s2}{\PYZdq{}} \PY{o}{+} \PY{l+s+sa}{f}\PY{l+s+s2}{\PYZdq{}}\PY{l+s+si}{\PYZob{}}\PY{n}{\PYZus{}}\PY{l+s+si}{:}\PY{l+s+s2}{.0f}\PY{l+s+si}{\PYZcb{}}\PY{l+s+s2}{°}\PY{l+s+s2}{\PYZdq{}}\PY{o}{+}\PY{l+s+sa}{r}\PY{l+s+s2}{\PYZdq{}}\PY{l+s+s2}{, \PYZdl{}h=\PYZdl{}}\PY{l+s+s2}{\PYZdq{}}\PY{o}{+}\PY{l+s+sa}{f}\PY{l+s+s2}{\PYZdq{}}\PY{l+s+si}{\PYZob{}}\PY{n}{\PYZus{}\PYZus{}}\PY{l+s+si}{:}\PY{l+s+s2}{.0f}\PY{l+s+si}{\PYZcb{}}\PY{l+s+s2}{\PYZdq{}}\PY{p}{,} \PY{n}{c}\PY{o}{=}\PY{n}{cmap2}\PY{o}{.}\PY{n}{to\PYZus{}rgba}\PY{p}{(}\PY{n}{\PYZus{}} \PY{o}{+} \PY{l+m+mi}{1}\PY{p}{)}\PY{p}{,} \PY{n}{linestyle}\PY{o}{=}\PY{n}{selectstyle}\PY{p}{[}\PY{n}{\PYZus{}\PYZus{}}\PY{o}{\PYZhy{}}\PY{l+m+mi}{1}\PY{p}{]}\PY{p}{)}

\PY{n}{plt}\PY{o}{.}\PY{n}{legend}\PY{p}{(}\PY{p}{)}
\PY{n}{plt}\PY{o}{.}\PY{n}{xlim}\PY{p}{(}\PY{o}{\PYZhy{}}\PY{l+m+mi}{1}\PY{p}{,} \PY{l+m+mi}{91}\PY{p}{)}
\PY{n}{plt}\PY{o}{.}\PY{n}{ylim}\PY{p}{(}\PY{l+m+mi}{0}\PY{p}{,} \PY{l+m+mi}{41}\PY{p}{)}
\PY{n}{plt}\PY{o}{.}\PY{n}{title}\PY{p}{(}\PY{l+s+s2}{\PYZdq{}}\PY{l+s+s2}{Posição vertical em função da horizontal}\PY{l+s+s2}{\PYZdq{}}\PY{p}{)}
\PY{n}{plt}\PY{o}{.}\PY{n}{xlabel}\PY{p}{(}\PY{l+s+s2}{\PYZdq{}}\PY{l+s+s2}{Horizontal (m)}\PY{l+s+s2}{\PYZdq{}}\PY{p}{)}
\PY{n}{plt}\PY{o}{.}\PY{n}{ylabel}\PY{p}{(}\PY{l+s+s2}{\PYZdq{}}\PY{l+s+s2}{Vertical (m)}\PY{l+s+s2}{\PYZdq{}}\PY{p}{)}
\end{Verbatim}
\end{tcolorbox}

            \begin{tcolorbox}[breakable, size=fbox, boxrule=.5pt, pad at break*=1mm, opacityfill=0]
\prompt{Out}{outcolor}{23}{\boxspacing}
\begin{Verbatim}[commandchars=\\\{\}]
Text(0, 0.5, 'Vertical (m)')
\end{Verbatim}
\end{tcolorbox}
        
    \begin{center}
    \adjustimage{max size={0.9\linewidth}{0.9\paperheight}}{Relatorio_files/Relatorio_48_1.png}
    \end{center}
    { \hspace*{\fill} \\}
    
    Note que, quando maior for \(h\) maior será a influência da resistência
do ar, pois a exponencial diminui mais lentamente por isso, o alcance é
maior para menores valores de \(h\). Além disso, ao cair, o projétil
terá uma resistência do ar crescente, o que faz o projétil cair mais
lentamente e desacelerar no eixo \(x\) mais rapidamente. Note que, em
alguns casos, o projétil fica sem velocidade na horizontal antes mesmo
de tocar o chão.


    % Add a bibliography block to the postdoc
    
    
    
\end{document}
